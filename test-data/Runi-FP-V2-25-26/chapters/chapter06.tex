% =============================================================================
% Chapter 6: Final Project Implementation
% מימוש פרויקט הגמר
% Based on original Chapter 8
% =============================================================================

\documentclass[../master/main.tex]{subfiles}

\begin{document}

\setcounter{chapter}{5}
\hebrewchapter{מימוש פרויקט הגמר}
\hebrewchapterlabel{chap:implementation}

\par\needspace{5\baselineskip}
\hebrewsection{מבוא: מהתיאוריה למעשה}

לאורך הספר עברנו מסע בעולם סוכני הבינה המלאכותית וליגת \qtwentyone{}. כעת הגיע הזמן לרכז את כל הידע לכדי מימוש מעשי. פרק זה מספק הנחיות מפורטות למימוש סוכן שחקן וסוכן שופט.

\par\needspace{4\baselineskip}
\hebrewsubsection{מטרות הפרק}

בסיום פרק זה, תבינו:
\begin{itemize}
    \item את הגדרת סביבת הפיתוח הנדרשת
    \item את מבנה סוכן השחקן ואחריותו
    \item את מבנה סוכן השופט ואחריותו
    \item את קבצי התצורה הנדרשים
    \item את אסטרטגיות הבדיקה המומלצות
\end{itemize}

% =============================================================================
\par\needspace{5\baselineskip}
\hebrewsection{סביבת הפיתוח}
\label{sec:dev-environment}
% =============================================================================

\par\needspace{4\baselineskip}
\hebrewsubsection{דרישות מערכת}

\begin{fancytable}{lH}{דרישות סביבה}
\label{tab:environment-requirements}
רכיב & דרישה \\
שפת תכנות & \en{Python} מומלץ, אך כל שפה אחרת מותרת \\
\en{Gmail API} & \num{2} חשבונות \en{Gmail} עם הרשאות \en{OAuth 2.0} (ראו נספח~א' ונספח~ה') \\
\en{CLI Terminal} & \en{Anthropic} או \en{Gemini} מומלץ, אך כל \en{LLM CLI} אחר אפשרי \\
\en{Git} & לניהול גרסאות \\
\en{PostgreSQL} & הקמת \num{2} מסדי נתונים, אחד לכל סוכן \en{AI} \\
\end{fancytable}

\par\needspace{4\baselineskip}
\hebrewsubsection{התקנת ספריות}

\begin{english}
\begin{pythonbox}[\hebtitle{התקנת ספריות נדרשות}]
# requirements.txt
google-api-python-client>=2.100.0
google-auth-httplib2>=0.1.0
google-auth-oauthlib>=1.0.0
anthropic>=0.20.0  # or openai>=1.0.0
python-dotenv>=1.0.0
aiohttp>=3.9.0
psycopg2-binary>=2.9.0  # optional, for PostgreSQL
\end{pythonbox}
\end{english}

\par\needspace{4\baselineskip}
\hebrewsubsection{מבנה הפרויקט המומלץ}

\begin{english}
\begin{pythonbox}[\hebtitle{מבנה ספריות הפרויקט}]
q21g-agents/
├── src/
│   ├── player/
│   │   ├── __init__.py
│   │   ├── player_agent.py      # Main player logic
│   │   ├── question_generator.py # Question generation
│   │   └── guess_strategy.py    # Guessing strategy
│   ├── referee/
│   │   ├── __init__.py
│   │   ├── referee_agent.py     # Main referee logic
│   │   ├── paragraph_selector.py # Paragraph selection
│   │   └── scorer.py            # Scoring logic
│   ├── shared/
│   │   ├── __init__.py
│   │   ├── gmail_client.py      # Gmail API wrapper
│   │   ├── protocol.py          # Message types
│   │   └── config.py            # Configuration
│   └── main.py                  # Entry point
├── config/
│   ├── credentials.json         # Gmail OAuth credentials
│   ├── player.env               # Player configuration
│   └── referee.env              # Referee configuration
├── data/
│   └── course_materials/        # Course materials PDFs
├── tests/
│   ├── test_player.py
│   ├── test_referee.py
│   └── test_protocol.py
├── requirements.txt
└── README.md
\end{pythonbox}
\end{english}

% =============================================================================
\par\needspace{5\baselineskip}
\hebrewsection{מימוש סוכן השחקן}
\label{sec:player-implementation}
% =============================================================================

\par\needspace{4\baselineskip}
\hebrewsubsection{מחזור החיים של השחקן}

\begin{enumerate}
    \item \textbf{אתחול} --- טעינת תצורה וחיבור ל-\gmailapi{}
    \item \textbf{רישום} --- שליחת \en{LEAGUE\_REGISTER\_REQUEST}
    \item \textbf{המתנה} --- מעקב אחר הודעות נכנסות
    \item \textbf{משחק} --- תגובה להזמנות וניהול שלבי המשחק
    \item \textbf{סיום} --- סגירת חיבורים ושמירת לוגים
\end{enumerate}

\par\needspace{4\baselineskip}
\hebrewsubsection{רצף הודעות שחקן מלא}

\begin{fancytable}{HlH}{הודעות שחקן - רישום ועונה}
\label{tab:player-messages-registration}
כיוון & סוג הודעה & תיאור \\
שליחה & \en{LEAGUE\_REGISTER\_REQUEST} & רישום לליגה \\
קבלה & \en{RESPONSE\_LEAGUE\_REGISTER} & אישור רישום + טבלת משתמשים \\
קבלה & \en{BROADCAST\_SEASON\_START} & התחלת עונה \\
שליחה & \en{SEASON\_REGISTRATION\_REQUEST} & רישום לעונה \\
קבלה & \en{BROADCAST\_ASSIGNMENT\_TABLE} & טבלת הקצאות \\
\end{fancytable}

\begin{fancytable}{HlH}{הודעות שחקן - משחק}
\label{tab:player-messages-game}
כיוון & סוג הודעה & תיאור \\
קבלה & \en{GAME\_INVITATION} & הזמנה למשחק \\
שליחה & \en{GAME\_JOIN\_ACK} & אישור הצטרפות (תוך \en{120s}) \\
קבלה & \en{Q21\_WARMUP\_CALL} & רמזים ראשוניים \\
שליחה & \en{Q21\_WARMUP\_RESPONSE} & אישור קבלת רמזים \\
קבלה & \en{Q21\_QUESTIONS\_CALL} & בקשת שאלות \\
שליחה & \en{Q21\_QUESTIONS\_RESPONSE} & \num{20} שאלות (תוך \en{300s}) \\
קבלה & \en{Q21\_ANSWERS\_CALL} & תשובות לשאלות \\
שליחה & \en{Q21\_FINAL\_GUESS} & ניחוש סופי \\
קבלה & \en{GAME\_OVER} & תוצאות המשחק \\
\end{fancytable}

\par\needspace{4\baselineskip}
\hebrewsubsection{מודול יצירת שאלות}

\begin{english}
\begin{pythonbox}[\hebtitle{דוגמה ליצירת שאלות}]
from anthropic import Anthropic

class QuestionGenerator:
    """Generates 20 multiple-choice questions based on hint."""

    def __init__(self, model: str = "claude-sonnet-4-20250514"):
        self.client = Anthropic()
        self.model = model

    async def generate_questions(
        self,
        description: str,
        association_topic: str
    ) -> list[dict]:
        """Generate 20 questions based on hint."""
        prompt = f"""
        Based on this description: "{description}"
        And association topic: "{association_topic}"

        Generate 20 strategic multiple-choice questions
        to identify the source paragraph.
        Each question should have 4 options.
        """

        response = await self.client.messages.create(
            model=self.model,
            max_tokens=4096,
            messages=[{"role": "user", "content": prompt}]
        )

        return self._parse_questions(response.content[0].text)
\end{pythonbox}
\end{english}

\par\needspace{10\baselineskip}
\hebrewsubsection{אסטרטגיית הניחוש}

\needspace{8\baselineskip}
\begin{implementationbox}
\textbf{טיפים לאסטרטגיית ניחוש:}
\begin{itemize}
    \item נתחו את התשובות לזיהוי דפוסים
    \item חפשו בחומרי הקורס פסקאות תואמות
    \item שקללו את הביטחון בכל ניחוש
    \item זכרו: המילה האסוציאטיבית שווה \num{30}\% מהציון!
\end{itemize}
\end{implementationbox}

% =============================================================================
\par\needspace{5\baselineskip}
\hebrewsection{מימוש סוכן השופט}
\label{sec:referee-implementation}
% =============================================================================

\par\needspace{4\baselineskip}
\hebrewsubsection{מחזור החיים של השופט}

\begin{enumerate}
    \item \textbf{אתחול} --- טעינת חומרי הקורס
    \item \textbf{רישום} --- שליחת \en{REFEREE\_REGISTER\_REQUEST}
    \item \textbf{הקצאה} --- קבלת טבלת הקצאות
    \item \textbf{ניהול משחק} --- הזמנה, רמזים, תשובות, ציונים
    \item \textbf{דיווח} --- שליחת \en{GAME\_RESULT}
\end{enumerate}

\par\needspace{4\baselineskip}
\hebrewsubsection{רצף הודעות שופט מלא}

\begin{fancytable}{HlH}{הודעות שופט - רישום}
\label{tab:referee-messages-registration}
כיוון & סוג הודעה & תיאור \\
שליחה & \en{REFEREE\_REGISTER\_REQUEST} & רישום כשופט \\
קבלה & \en{RESPONSE\_REFEREE\_REGISTER} & אישור רישום \\
קבלה & \en{BROADCAST\_ASSIGNMENT\_TABLE} & משחקים שהוקצו \\
\end{fancytable}

\begin{fancytable}{HlH}{הודעות שופט - ניהול משחק}
\label{tab:referee-messages-game}
כיוון & סוג הודעה & תיאור \\
שליחה & \en{GAME\_INVITATION} & הזמנת שחקנים \\
קבלה & \en{GAME\_JOIN\_ACK} & אישורי הצטרפות \\
שליחה & \en{Q21\_WARMUP\_CALL} & שליחת רמזים \\
קבלה & \en{Q21\_WARMUP\_RESPONSE} & אישור קבלה \\
שליחה & \en{Q21\_QUESTIONS\_CALL} & בקשת שאלות \\
קבלה & \en{Q21\_QUESTIONS\_RESPONSE} & קבלת שאלות \\
שליחה & \en{Q21\_ANSWERS\_CALL} & שליחת תשובות \\
קבלה & \en{Q21\_FINAL\_GUESS} & קבלת ניחושים \\
שליחה & \en{GAME\_OVER} & תוצאות לשחקנים \\
שליחה & \en{GAME\_RESULT} & דיווח למנהל \\
\end{fancytable}

\begin{fancytable}{HlH}{הודעות שופט - כוח עליון}
\label{tab:referee-messages-fm}
כיוון & סוג הודעה & תיאור \\
שליחה & \en{FORCE\_MAJEURE\_REQUEST} & דיווח על אירוע חריג \\
קבלה & \en{FORCE\_MAJEURE\_DECISION} & החלטת המנהל \\
קבלה & \en{FORCE\_MAJEURE\_RESULT} & תוצאה סופית \\
\end{fancytable}

\par\needspace{4\baselineskip}
\hebrewsubsection{בחירת פסקה ויצירת רמזים}

\begin{english}
\begin{pythonbox}[\hebtitle{בחירת פסקה}]
import random

class ParagraphSelector:
    """Selects paragraph and creates hints."""

    def __init__(self, materials_path: str):
        self.materials = self._load_materials(materials_path)

    def select_paragraph(self) -> dict:
        """Select a random paragraph from course materials."""
        lecture = random.choice(self.materials)
        paragraph = random.choice(lecture["paragraphs"])

        return {
            "lecture_id": lecture["id"],
            "paragraph_id": paragraph["id"],
            "text": paragraph["text"],
            "opening_sentence": paragraph["text"].split(".")[0] + "."
        }

    def create_hints(self, paragraph: dict) -> dict:
        """Create description and association for paragraph."""
        # Use LLM to generate hints
        # IMPORTANT: description must NOT contain words from paragraph!
        return {
            "secret_title": "...",  # Max 5 words
            "description": "...",    # Max 15 words, no overlap
            "association_word": "...",
            "association_topic": "..."
        }
\end{pythonbox}
\end{english}

\par\needspace{4\baselineskip}
\hebrewsubsection{חישוב ציונים}

\begin{english}
\begin{pythonbox}[\hebtitle{חישוב ציון סיבוב}]
class Scorer:
    """Scores player guesses against original paragraph."""

    def score_guess(
        self,
        original: dict,
        guess: dict
    ) -> dict:
        """Calculate score (0-100) with breakdown."""

        # 50% - Opening sentence accuracy
        sentence_score = self._compare_sentences(
            original["opening_sentence"],
            guess["opening_sentence"]
        )

        # 20% - Sentence reasoning quality
        reasoning_score = self._evaluate_reasoning(
            guess["sentence_reasoning"]
        )

        # 20% - Association word accuracy
        association_score = self._compare_association(
            original["association_word"],
            guess["association_word"],
            guess["association_variations"]
        )

        # 10% - Association reasoning quality
        assoc_reasoning_score = self._evaluate_reasoning(
            guess["association_reasoning"]
        )

        total = (
            sentence_score * 0.5 +
            reasoning_score * 0.2 +
            association_score * 0.2 +
            assoc_reasoning_score * 0.1
        )

        return {
            "total": round(total, 2),
            "breakdown": {
                "opening_sentence": sentence_score,
                "sentence_reasoning": reasoning_score,
                "association_word": association_score,
                "association_reasoning": assoc_reasoning_score
            }
        }
\end{pythonbox}
\end{english}

% =============================================================================
\par\needspace{5\baselineskip}
\hebrewsection{קבצי תצורה}
\label{sec:config-files}
% =============================================================================

\par\needspace{4\baselineskip}
\hebrewsubsection{תצורת השחקן}

\begin{english}
\begin{pythonbox}[\hebtitle{player.env}]
# Player Configuration
PLAYER_EMAIL=MyGroup.player@gmail.com
PLAYER_NAME=MyGroup Player
LEAGUE_MANAGER_EMAIL=bitalevi100@gmail.com
LOG_SERVER_EMAIL=beit.halevi.700@gmail.com

# LLM Configuration
LLM_PROVIDER=anthropic
LLM_MODEL=claude-sonnet-4-20250514
LLM_API_KEY=sk-ant-...

# Gmail API
GMAIL_CREDENTIALS_PATH=./config/credentials.json
GMAIL_TOKEN_PATH=./config/token.json

# Polling interval (seconds)
POLL_INTERVAL=30
\end{pythonbox}
\end{english}

\par\needspace{4\baselineskip}
\hebrewsubsection{תצורת השופט}

\begin{english}
\begin{pythonbox}[\hebtitle{referee.env}]
# Referee Configuration
REFEREE_EMAIL=MyGroup.referee@gmail.com
REFEREE_NAME=MyGroup Referee
LEAGUE_MANAGER_EMAIL=bitalevi100@gmail.com
LOG_SERVER_EMAIL=beit.halevi.700@gmail.com

# Course Materials
MATERIALS_PATH=./data/course_materials/

# LLM Configuration
LLM_PROVIDER=anthropic
LLM_MODEL=claude-sonnet-4-20250514
LLM_API_KEY=sk-ant-...

# Gmail API
GMAIL_CREDENTIALS_PATH=./config/credentials.json
GMAIL_TOKEN_PATH=./config/token.json

# Polling interval (seconds)
POLL_INTERVAL=30
\end{pythonbox}
\end{english}

% =============================================================================
\par\needspace{5\baselineskip}
\hebrewsection{בדיקות}
\label{sec:testing}
% =============================================================================

\par\needspace{4\baselineskip}
\hebrewsubsection{אסטרטגיות בדיקה}

\begin{enumerate}
    \item \textbf{בדיקות יחידה} --- בדיקת פונקציות בודדות
    \item \textbf{בדיקות אינטגרציה} --- בדיקת תקשורת עם \gmailapi{}
    \item \textbf{בדיקות קצה-לקצה} --- משחק מלא בין סוכנים
    \item \textbf{בדיקות מול עצמכם} --- שחקו נגד השופט שלכם
\end{enumerate}

\par\needspace{4\baselineskip}
\hebrewsubsection{רשימת בדיקות חובה}

\begin{fancytable}{lH}{רשימת בדיקות חובה}
\label{tab:testing-checklist}
בדיקה & תיאור \\
רישום & הסוכן נרשם בהצלחה ומקבל מזהה \\
תגובה להזמנה & השחקן מגיב להזמנת משחק בזמן \\
יצירת שאלות & השחקן מייצר \num{20} שאלות תקינות \\
ניחוש סופי & השחקן שולח ניחוש עם כל הרכיבים \\
חישוב ציון & השופט מחזיר ציון תקין עם פירוט \\
\en{CC} לשרת הלוג & כל הודעה כוללת \en{CC} \\
עמידה במועדים & תגובות בתוך חלונות הזמן \\
\en{GateKeeper} & מגבלות קצב ומכסות מיושמות \\
טיפול בשידורים & תגובה לשידורים שדורשים תגובה \\
בקשת הארכה & יכולת לבקש הארכה לפני דדליין \\
מעקב הודעות & טבלת מעקב עם סטטוסים \\
\end{fancytable}

% =============================================================================
\par\needspace{5\baselineskip}
\hebrewsection{מימוש טבלת מעקב הודעות}
\label{sec:user-message-table}
% =============================================================================

הסוכן חייב לעקוב אחר כל הודעה שנשלחה ושהתקבלה. הטבלה מאפשרת זיהוי תגובות חסרות ומניעת שליחה כפולה.

\par\needspace{4\baselineskip}
\hebrewsubsection{מבנה הטבלה}

\begin{fancytable}{llH}{מבנה טבלת מעקב הודעות}
\label{tab:user-message-table}
שדה & סוג & תיאור \\
\en{message\_id} & \en{UUID} & מזהה ייחודי של ההודעה \\
\en{correlation\_id} & \en{UUID} & מזהה לקישור בקשה-תגובה \\
\en{message\_type} & \en{VARCHAR} & סוג ההודעה \\
\en{direction} & \en{ENUM} & \en{SENT} או \en{RECEIVED} \\
\en{status} & \en{ENUM} & \en{OPEN}, \en{CLOSED}, \en{REJECTED} \\
\en{deadline} & \en{TIMESTAMP} & מועד אחרון לתגובה \\
\en{created\_at} & \en{TIMESTAMP} & זמן יצירה \\
\en{closed\_at} & \en{TIMESTAMP} & זמן סגירה (אם רלוונטי) \\
\end{fancytable}

\par\needspace{4\baselineskip}
\hebrewsubsection{מימוש מעקב}

\begin{english}
\begin{pythonbox}[\hebtitle{מחלקת מעקב הודעות}]
from datetime import datetime, timedelta
from enum import Enum
from dataclasses import dataclass
from typing import Optional

class MessageStatus(Enum):
    OPEN = "OPEN"       # Waiting for response
    CLOSED = "CLOSED"   # Response received
    REJECTED = "REJECTED"  # Deadline passed

@dataclass
class TrackedMessage:
    message_id: str
    correlation_id: str
    message_type: str
    deadline: Optional[datetime]
    status: MessageStatus = MessageStatus.OPEN

class MessageTracker:
    def __init__(self):
        self.messages: dict[str, TrackedMessage] = {}

    def track_sent(self, msg: dict, deadline_seconds: int = None):
        """Track outgoing message with optional deadline."""
        deadline = None
        if deadline_seconds:
            deadline = datetime.now() + timedelta(seconds=deadline_seconds)
        tracked = TrackedMessage(
            message_id=msg["message_id"],
            correlation_id=msg.get("correlation_id"),
            message_type=msg["message_type"],
            deadline=deadline
        )
        self.messages[msg["correlation_id"]] = tracked

    def close_by_correlation(self, correlation_id: str):
        """Mark message as closed when response received."""
        if correlation_id in self.messages:
            self.messages[correlation_id].status = MessageStatus.CLOSED
\end{pythonbox}
\end{english}

% =============================================================================
\par\needspace{5\baselineskip}
\hebrewsection{מימוש מטפל שידורים}
\label{sec:broadcast-handler-impl}
% =============================================================================

השרת שולח \num{11} סוגי שידורים שונים. חלקם דורשים תגובה ויש להם דדליין.

\par\needspace{4\baselineskip}
\hebrewsubsection{זיהוי שידורים}

\begin{english}
\begin{pythonbox}[\hebtitle{זיהוי סוג ההודעה}]
# Broadcast types that require response
RESPONSE_REQUIRED = {
    "BROADCAST_CONNECTIVITY_TEST": 60,     # 60 seconds
    "BROADCAST_KEEP_ALIVE": 300,           # 5 minutes
    "BROADCAST_CRITICAL_SYSTEM": 120,      # 2 minutes
    "BROADCAST_CRITICAL_SECURITY": 120,    # 2 minutes
}

# Broadcast types - no response needed
NO_RESPONSE_NEEDED = {
    "BROADCAST_SEASON_START",
    "BROADCAST_ROUND_START",
    "BROADCAST_ROUND_END",
    "BROADCAST_SEASON_END",
    "BROADCAST_ASSIGNMENT_TABLE",
    "BROADCAST_STANDINGS_UPDATE",
    "BROADCAST_MAINTENANCE_NOTICE",
}

def get_response_deadline(message_type: str) -> int | None:
    """Get deadline in seconds, or None if no response needed."""
    return RESPONSE_REQUIRED.get(message_type)
\end{pythonbox}
\end{english}

\par\needspace{4\baselineskip}
\hebrewsubsection{טיפול בשידורים}

\begin{english}
\begin{pythonbox}[\hebtitle{מטפל שידורים}]
class BroadcastHandler:
    def __init__(self, email_client, tracker: MessageTracker):
        self.email_client = email_client
        self.tracker = tracker

    async def handle_broadcast(self, msg: dict):
        """Process incoming broadcast message."""
        msg_type = msg["message_type"]
        deadline_seconds = get_response_deadline(msg_type)

        if deadline_seconds:
            # Response required - track and respond
            self.tracker.track_sent(msg, deadline_seconds)
            await self._send_broadcast_response(msg, deadline_seconds)
        else:
            # Informational only - log and process
            self._process_informational(msg)

    async def _send_broadcast_response(self, msg: dict, deadline: int):
        """Send acknowledgment response to broadcast."""
        response = {
            "message_type": f"RESPONSE_{msg['message_type']}",
            "correlation_id": msg["message_id"],
            "payload": {"status": "ACKNOWLEDGED"}
        }
        await self.email_client.send(response)
\end{pythonbox}
\end{english}

% =============================================================================
\par\needspace{5\baselineskip}
\hebrewsection{מימוש בקשת הארכה}
\label{sec:extension-handler-impl}
% =============================================================================

כאשר הסוכן צופה שלא יספיק לעמוד בדדליין, הוא יכול לבקש הארכה.

\par\needspace{4\baselineskip}
\hebrewsubsection{מתי לבקש הארכה}

\begin{warningbox}[\hebtitle{כלל הזהב}]
בקשת הארכה חייבת להישלח \textbf{לפני} שפג תוקף הדדליין. בקשה מאוחרת תידחה!
\end{warningbox}

\par\needspace{4\baselineskip}
\hebrewsubsection{מימוש מבקש הארכה}

\begin{english}
\begin{pythonbox}[\hebtitle{מנהל הארכות}]
class ExtensionManager:
    def __init__(self, email_client, tracker: MessageTracker):
        self.email_client = email_client
        self.tracker = tracker
        self.pending_extensions: dict[str, bool] = {}

    async def request_if_needed(
        self,
        original_msg: dict,
        deadline: datetime,
        estimated_processing: float
    ) -> bool:
        """Request extension if we won't make deadline."""
        time_remaining = (deadline - datetime.now()).total_seconds()
        buffer = 30  # Safety margin

        if time_remaining < estimated_processing + buffer:
            return await self._send_extension_request(original_msg)
        return True  # No extension needed

    async def _send_extension_request(self, original_msg: dict) -> bool:
        """Send extension request and wait for response."""
        request = {
            "message_type": "EXTENSION_REQUEST",
            "payload": {
                "original_message_id": original_msg["message_id"],
                "original_message_type": original_msg["message_type"],
                "reason": "Processing requires additional time",
                "requested_extension_seconds": 120
            }
        }
        correlation_id = request.get("correlation_id")
        self.pending_extensions[correlation_id] = None

        await self.email_client.send(request)
        # Wait for RESPONSE_EXTENSION...
        return await self._wait_for_response(correlation_id)

    def handle_extension_response(self, response: dict):
        """Process extension response from server."""
        correlation_id = response.get("correlation_id")
        if correlation_id in self.pending_extensions:
            status = response["payload"]["status"]
            self.pending_extensions[correlation_id] = (status == "APPROVED")
\end{pythonbox}
\end{english}

% =============================================================================
\par\needspace{5\baselineskip}
\hebrewsection{הרצת הליגה}
\label{sec:running-league}
% =============================================================================

\par\needspace{4\baselineskip}
\hebrewsubsection{שלבי ההפעלה}

\begin{enumerate}
    \item הגדירו את קבצי התצורה
    \item הפעילו את סוכן השופט: \en{\texttt{python main.py referee}}
    \item הפעילו את סוכן השחקן: \en{\texttt{python main.py player}}
    \item הסוכנים יירשמו אוטומטית ויחכו להודעות
    \item בצעו בדיקות מול עצמכם לפני הליגה הרשמית
\end{enumerate}

\par\needspace{10\baselineskip}
\hebrewsubsection{טיפים להצלחה}

\needspace{10\baselineskip}
\begin{implementationbox}
\textbf{טיפים להצלחה בליגה:}
\begin{itemize}
    \item \textbf{לוגים} --- תעדו כל הודעה נכנסת ויוצאת
    \item \textbf{מועדים} --- הוסיפו מרווח ביטחון (אל תחכו לשנייה האחרונה)
    \item \textbf{טיפול בשגיאות} --- טפלו בכל חריגה, כולל שגיאת \num{429} (\en{Rate Limit})
    \item \textbf{מצב יציב} --- שמרו מצב למקרה של הפעלה מחדש
    \item \textbf{תרגול} --- שחקו מול עצמכם שוב ושוב
    \item \textbf{מגבלות \en{API}} --- הכירו את מגבלות \en{Gmail API} (ראו נספח~ה')
\end{itemize}
\end{implementationbox}

% =============================================================================
\par\needspace{5\baselineskip}
\hebrewsection{סיכום}
% =============================================================================

פרק זה סיפק הנחיות מעשיות למימוש:

\begin{itemize}
    \item הגדרת סביבת פיתוח עם \en{Python 3.10+} ו-\gmailapi{}
    \item מבנה פרויקט מומלץ עם הפרדה לשחקן ושופט
    \item מימוש סוכן שחקן עם יצירת שאלות ואסטרטגיית ניחוש
    \item מימוש סוכן שופט עם בחירת פסקה וחישוב ציונים
    \item קבצי תצורה לשני הסוכנים
    \item אסטרטגיות בדיקה ורשימת בדיקות חובה
\end{itemize}

בהצלחה בפרויקט הגמר!

\end{document}
