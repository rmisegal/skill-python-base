% =============================================================================
% Appendix H: League Manager Architecture Reference
% עקרונות ארכיטקטוניים של מנהל הליגה
% Reference material for the League Manager server architecture
% =============================================================================

\documentclass[../master/main.tex]{subfiles}

\begin{document}

\chapter*{נספח ח': עקרונות ארכיטקטוניים של מנהל הליגה}
\addcontentsline{toc}{chapter}{נספח ח': עקרונות ארכיטקטוניים של מנהל הליגה}
\renewcommand{\thehebrewsection}{ח.\arabic{hebrewsection}}
\setcounter{hebrewsection}{0}

\par\needspace{5\baselineskip}
\hebrewsection{מבוא}

נספח זה מתאר את העקרונות הארכיטקטוניים של שרת מנהל הליגה. חומר זה מיועד כרקע להבנת המערכת הכוללת, ומהווה בסיס להשוואה עם ארכיטקטורת הסוכנים (פרק~\ref{chap:architecture-principles}).

% =============================================================================
\par\needspace{5\baselineskip}
\hebrewsection{הפרדת נתונים סטטיים ודינמיים}
\label{sec:lgm-static-dynamic}
% =============================================================================

\par\needspace{4\baselineskip}
\hebrewsubsection{העיקרון המנחה}

מנהל הליגה מפריד בין שני סוגי נתונים עיקריים:

\begin{enumerate}
    \item \textbf{נתונים סטטיים} --- מידע שאינו משתנה במהלך העונה
    \item \textbf{נתונים דינמיים} --- מידע שמתעדכן בזמן אמת
\end{enumerate}

\par\needspace{4\baselineskip}
\hebrewsubsection{טבלת קבוצות סטודנטים (\en{student\_groups})}

טבלת \en{student\_groups} מכילה את המידע הסטטי של כל קבוצה:

\begin{fancytable}{lHH}{נתונים סטטיים בטבלת קבוצות}
\label{tab:lgm-static-data-fields}
שדה & סוג & תיאור \\
\en{group\_id} & \en{VARCHAR} & מזהה קבוצה ייחודי (\en{G001}--\en{G030}) \\
\en{group\_name} & \en{VARCHAR(8)} & שם הקבוצה (\num{8} תווים) \\
\en{student\_emails} & \en{JSON} & רשימת כתובות אימייל של הסטודנטים \\
\en{player\_github\_repo} & \en{URL} & קישור לריפוזיטורי השחקן \\
\en{referee\_github\_repo} & \en{URL} & קישור לריפוזיטורי השופט \\
\en{registered\_at} & \en{TIMESTAMP} & זמן הרשמה למערכת \\
\end{fancytable}

\par\needspace{4\baselineskip}
\hebrewsubsection{נתונים דינמיים --- שחקנים ושופטים}

טבלאות השחקנים (\en{players}) והשופטים (\en{referees}) מכילות נתונים דינמיים:

\begin{itemize}
    \item \textbf{מצב נוכחי} --- האם הסוכן פעיל, מושהה, או כבוי
    \item \textbf{סטטיסטיקות עונה} --- משחקים, ניצחונות, נקודות
    \item \textbf{מזהה עונה נוכחית} --- לאיזו עונה הסוכן רשום
\end{itemize}

% =============================================================================
\par\needspace{5\baselineskip}
\hebrewsection{תבנית ה-\en{Gatekeeper} של מנהל הליגה}
\label{sec:lgm-gatekeeper}
% =============================================================================

\par\needspace{4\baselineskip}
\hebrewsubsection{הבעיה}

\en{Gmail API} מטיל מגבלות קשיחות על כמות ההודעות והקריאות. חריגה מהמגבלות גורמת לחסימה זמנית של החשבון.

\par\needspace{4\baselineskip}
\hebrewsubsection{הפתרון --- שכבת הגנה}

ה-\en{Gatekeeper} של מנהל הליגה הוא תבנית \en{Facade} עם שלושה רכיבים:

\begin{fancytable}{lHH}{רכיבי ה-\en{Gatekeeper} של מנהל הליגה}
\label{tab:lgm-gatekeeper-components}
רכיב & תפקיד & פעולה בחריגה \\
\en{QuotaManager} & מעקב אחר מכסה יומית & עצירת שליחות עד חידוש \\
\en{DOSDetector} & זיהוי דפוסים חשודים & חסימה זמנית והתראה \\
\en{RateLimiter} & הגבלת קצב שליחה & השהיה אוטומטית \\
\end{fancytable}

\needspace{10\baselineskip}
\begin{notebox}[\hebtitle{מגבלה תפעולית}]
מנהל הליגה מגדיר מגבלה תפעולית של \num{400} הודעות ליום (מתוך מכסת \num{500}) כדי לשמור על מרווח ביטחון.
\end{notebox}

% =============================================================================
\par\needspace{5\baselineskip}
\hebrewsection{מערכת השעונים של מנהל הליגה}
\label{sec:lgm-clocks}
% =============================================================================

מנהל הליגה מנהל חמישה סוגי שעונים:

\begin{fancytable}{lR{4cm}R{5.5cm}}{חמשת השעונים של מנהל הליגה}
\label{tab:lgm-five-clocks}
שעון & תפקיד & מצבים \\
\en{SeasonClock} & ניהול עונת הליגה & \en{SETUP, ACTIVE, COMPLETED} \\
\en{RegistrationClock} & חלון הרשמה & \en{CLOSED, OPEN, CLOSED} \\
\en{RoundClock} & מחזור משחקים & \en{SCHEDULED, IN\_PROGRESS, COMPLETED} \\
\en{MessageDeadline} & מועד תגובה להודעה & \en{PENDING, EXPIRED} \\
\en{CalendarWindow} & חלון זמנים למשחקים & \en{OUTSIDE, ACTIVE, GRACE} \\
\end{fancytable}

% =============================================================================
\par\needspace{5\baselineskip}
\hebrewsection{איחוד שידורים (\en{Broadcast Consolidation})}
\label{sec:lgm-broadcast-consolidation}
% =============================================================================

\par\needspace{4\baselineskip}
\hebrewsubsection{הבעיה}

שליחת הודעת שידור ל-\num{30} משתתפים כ-\num{30} אימיילים נפרדים צורכת \num{30} קריאות \en{API}.

\par\needspace{4\baselineskip}
\hebrewsubsection{הפתרון}

במקום לשלוח הודעות נפרדות, מנהל הליגה שולח הודעה \textbf{אחת} עם כל הנמענים בשדה \en{TO}. זה מפחית את צריכת ה-\en{API} ב-\num{97}\%.

\par\needspace{4\baselineskip}
\hebrewsubsection{מיפוי תגובות}

כאשר נמען מגיב להודעת שידור, המערכת מזהה את ההודעה המקורית באמצעות:

\begin{enumerate}
    \item \textbf{\en{broadcast\_id}} --- מזהה ייחודי בנושא האימייל
    \item \textbf{טבלת מיפוי} --- \en{broadcast\_transaction\_mapping}
    \item \textbf{שדה \en{In-Reply-To}} --- קישור לכותרת ההודעה המקורית
\end{enumerate}

% =============================================================================
\par\needspace{5\baselineskip}
\hebrewsection{הורשת הקשר (\en{Context Inheritance})}
\label{sec:lgm-context-inheritance}
% =============================================================================

כל הודעה במערכת נושאת שדות הקשר שמזהים את מיקומה בהיררכיה:

\begin{fancytable}{lHH}{שדות הקשר}
\label{tab:lgm-context-fields}
שדה & דוגמה & תיאור \\
\en{league\_id} & \en{Q21G\_2026} & מזהה הליגה \\
\en{season\_id} & \en{S01} & מזהה העונה \\
\en{round\_id} & \en{01} & מספר המחזור \\
\en{game\_id} & \en{0101001} & מזהה משחק מורכב \\
\end{fancytable}

\needspace{10\baselineskip}
\begin{protocolbox}[width=\textwidth]
\textbf{כלל ההורשה:} תגובה להודעה חייבת לכלול את אותם שדות הקשר שהופיעו בהודעה המקורית.
\end{protocolbox}

% =============================================================================
\par\needspace{5\baselineskip}
\hebrewsection{מכונות מצבים}
\label{sec:lgm-state-machines}
% =============================================================================

מנהל הליגה משתמש בתבניות מצב עקביות:

\begin{enumerate}
    \item \textbf{תבנית מחזור חיים} --- \en{PENDING → ACTIVE → COMPLETED}
    \item \textbf{תבנית רישום} --- \en{UNREGISTERED → REGISTERED → ACTIVE}
    \item \textbf{תבנית משחק} --- \en{SCHEDULED → IN\_PROGRESS → FINISHED}
    \item \textbf{תבנית בקשה} --- \en{PENDING → APPROVED/REJECTED}
\end{enumerate}

\par\needspace{4\baselineskip}
\hebrewsubsection{עקרונות מעבר מצבים}

\begin{itemize}
    \item \textbf{חד-כיווניות} --- מעברים מתקדמים בלבד (למעט \en{PAUSED})
    \item \textbf{אטומיות} --- מעבר מצב הוא פעולה אטומית
    \item \textbf{תיעוד} --- כל מעבר מצב נרשם עם חותמת זמן
    \item \textbf{טריגרים} --- כל מעבר מופעל על ידי אירוע מוגדר
\end{itemize}

% =============================================================================
\par\needspace{5\baselineskip}
\hebrewsection{סיכום}
% =============================================================================

נספח זה סיכם את העקרונות הארכיטקטוניים של מנהל הליגה:

\begin{itemize}
    \item \textbf{הפרדת נתונים} --- סטטיים מול דינמיים
    \item \textbf{תבנית \en{Gatekeeper}} --- הגנה על \en{Gmail API}
    \item \textbf{חמישה שעונים} --- ניהול זמנים ומועדים
    \item \textbf{איחוד שידורים} --- הפחתת קריאות \en{API}
    \item \textbf{הורשת הקשר} --- העברת שדות הקשר בתגובות
    \item \textbf{מכונות מצבים} --- תבניות עקביות לניהול מצב
\end{itemize}

לפרטים על יישום עקרונות אלו בסוכן השחקן והשופט, ראו פרק~\ref{chap:architecture-principles}.

\end{document}
