% main.tex - Master Document
% כלי בינה מלאכותית בעסקים (AI Tools in Business)
% Authors: Dr. Yoram Segal & Prof. Eran Sheriff
% Compiler: LuaLaTeX

% preamble.tex - Shared preamble for AI Tools in Business Book
% Authors: Dr. Yoram Segal & Prof. Eran Sheriff
% Compiler: LuaLaTeX

\documentclass[12pt,a4paper,openright,twoside]{book}

%% ============================================
%% Engine and Encoding
%% ============================================
\usepackage{fontspec}
\usepackage{luatexbase}

%% ============================================
%% Language Support (Hebrew RTL + English LTR)
%% ============================================
\usepackage{polyglossia}
\setmainlanguage{hebrew}
\setotherlanguage{english}

% Hebrew Fonts
\newfontfamily\hebrewfont{David CLM}[Script=Hebrew]
\newfontfamily\hebrewfontsf{Miriam CLM}[Script=Hebrew]
\newfontfamily\hebrewfonttt{Miriam Mono CLM}[Script=Hebrew]

% English Fonts
\setmainfont{Times New Roman}
\setsansfont{Arial}
\setmonofont{Courier New}

%% ============================================
%% Page Layout
%% ============================================
\usepackage[
  a4paper,
  inner=3cm,
  outer=2.5cm,
  top=2.5cm,
  bottom=2.5cm,
  headheight=14pt
]{geometry}

% Prevent excessive vertical stretching between elements
% Book class uses \flushbottom by default, which stretches
% itemize/enumerate spacing to fill pages. Use \raggedbottom instead.
\raggedbottom

%% ============================================
%% Graphics and Colors
%% ============================================
\usepackage{graphicx}
\usepackage{xcolor}
\usepackage{tikz}
\usetikzlibrary{shapes,arrows,positioning,calc,fit,backgrounds,decorations.pathreplacing}
\usepackage{pgfplots}
\pgfplotsset{compat=1.18}

% Define book colors
\definecolor{chaptercolor}{RGB}{0,51,102}
\definecolor{sectioncolor}{RGB}{0,76,153}
\definecolor{examplecolor}{RGB}{230,242,255}
\definecolor{exercisecolor}{RGB}{255,248,220}
\definecolor{codebackground}{RGB}{245,245,245}
\definecolor{formulacolor}{RGB}{240,255,240}

%% ============================================
%% Math Packages
%% ============================================
\usepackage{amsmath}
\usepackage{amssymb}
\usepackage{amsthm}
\usepackage{mathtools}

%% ============================================
%% Tables
%% ============================================
\usepackage{booktabs}
\usepackage{tabularx}
\usepackage{longtable}
\usepackage{multirow}
\usepackage{array}

%% ============================================
%% Code Listings
%% ============================================
\usepackage{listings}
\usepackage{minted}

\lstset{
  basicstyle=\ttfamily\small,
  backgroundcolor=\color{codebackground},
  frame=single,
  breaklines=true,
  numbers=left,
  numberstyle=\tiny\color{gray},
  keywordstyle=\color{blue}\bfseries,
  commentstyle=\color{green!50!black},
  stringstyle=\color{red!60!black},
  showstringspaces=false,
  tabsize=4
}

% Python style
\lstdefinestyle{python}{
  language=Python,
  morekeywords={self,True,False,None,as,with,async,await}
}

% JSON style
\lstdefinestyle{json}{
  basicstyle=\ttfamily\small,
  stringstyle=\color{red!60!black},
  morestring=[b]",
  literate=
    *{:}{{{\color{blue}:}}}{1}
    {,}{{{\color{blue},}}}{1}
    {\{}{{{\color{blue}\{}}}{1}
    {\}}{{{\color{blue}\}}}}{1}
    {[}{{{\color{blue}[}}}{1}
    {]}{{{\color{blue}]}}}{1}
}

%% ============================================
%% Boxes and Environments (BiDi-safe tcolorbox)
%% Pattern: qa-BiDi-fix-tcolorbox - wrapper for RTL background overflow
%% ============================================
\usepackage{tcolorbox}
\tcbuselibrary{skins,breakable,theorems,listings}

% ============== INTERNAL BOXES (with @inner suffix) ==============

% Example Box - Internal
\newtcolorbox{examplebox@inner}[1][]{
  enhanced,
  breakable,
  colback=examplecolor,
  colframe=sectioncolor,
  fonttitle=\bfseries,
  title={\texthebrew{#1}},
  halign title=flush right,
  arc=3mm,
  boxrule=1pt
}

% Exercise Box - Internal
\newtcolorbox{exercisebox@inner}[1][]{
  enhanced,
  breakable,
  colback=exercisecolor,
  colframe=orange!70!black,
  fonttitle=\bfseries,
  title={\texthebrew{#1}},
  halign title=flush right,
  arc=3mm,
  boxrule=1pt
}

% Formula Box - Internal
\newtcolorbox{formulabox@inner}[1][]{
  enhanced,
  breakable,
  colback=formulacolor,
  colframe=green!50!black,
  fonttitle=\bfseries,
  title={\texthebrew{#1}},
  halign title=flush right,
  arc=2mm,
  boxrule=0.5pt
}

% Code Box - Internal
\newtcolorbox{codebox@inner}[1][]{
  enhanced,
  breakable,
  colback=codebackground,
  colframe=gray!50,
  fonttitle=\bfseries\ttfamily,
  title={\texthebrew{#1}},
  halign title=flush right,
  arc=2mm,
  boxrule=0.5pt
}

% Note Box - Internal
\newtcolorbox{notebox@inner}[1][]{
  enhanced,
  breakable,
  colback=yellow!10,
  colframe=yellow!50!black,
  fonttitle=\bfseries,
  title={\texthebrew{#1}},
  halign title=flush right,
  arc=2mm,
  boxrule=1pt
}

% ============== BiDi-SAFE WRAPPER ENVIRONMENTS ==============
% Force LTR for box drawing, RTL for content inside

\newenvironment{examplebox}[1][]
  {\begin{english}\begin{examplebox@inner}[#1]\selectlanguage{hebrew}}
  {\end{examplebox@inner}\end{english}}

\newenvironment{exercisebox}[1][]
  {\begin{english}\begin{exercisebox@inner}[#1]\selectlanguage{hebrew}}
  {\end{exercisebox@inner}\end{english}}

\newenvironment{formulabox}[1][]
  {\begin{english}\begin{formulabox@inner}[#1]\selectlanguage{hebrew}}
  {\end{formulabox@inner}\end{english}}

\newenvironment{codebox}[1][]
  {\begin{english}\begin{codebox@inner}[#1]\selectlanguage{hebrew}}
  {\end{codebox@inner}\end{english}}

\newenvironment{notebox}[1][]
  {\begin{english}\begin{notebox@inner}[#1]\selectlanguage{hebrew}}
  {\end{notebox@inner}\end{english}}

% Python code box with proper LTR direction
\newcommand{\pythonverbatimformat}{%
  \selectlanguage{english}%
  \textdir TLT\pardir TLT%
  \ttfamily\footnotesize%
}

\newtcblisting{pythonbox*}[1][]{
  listing engine=listings,
  listing only,
  colback=codebackground,
  colframe=codebackground,
  arc=2pt,
  boxrule=0pt,
  left=8pt,
  right=8pt,
  top=8pt,
  bottom=8pt,
  fonttitle=\bfseries,
  coltitle=black,
  title=#1,
  before upper={\selectlanguage{english}\textdir TLT\pardir TLT},
  listing options={
    basicstyle=\pythonverbatimformat,
    tabsize=4,
    breaklines=true,
    showspaces=false,
    showtabs=false,
    language=Python
  }
}

% Hebrew title for pythonbox in RTL context
\newcommand{\hebtitle}[1]{{\texthebrew{#1}}}

%% ============================================
%% Chapter and Section Styling
%% ============================================
\usepackage{titlesec}

\titleformat{\chapter}[display]
  {\normalfont\huge\bfseries\color{chaptercolor}}
  {\chaptertitlename\ \textenglish{\thechapter}}{20pt}{\Huge}

\titleformat{\section}
  {\normalfont\Large\bfseries\color{sectioncolor}}
  {\textenglish{\thesection}}{1em}{}

\titleformat{\subsection}
  {\normalfont\large\bfseries}
  {\textenglish{\thesubsection}}{1em}{}

%% ============================================
%% Headers and Footers
%% ============================================
\usepackage{fancyhdr}
\pagestyle{fancy}
\fancyhf{}
\fancyhead[RO]{\leftmark}
\fancyhead[LE]{\rightmark}
\fancyfoot[C]{\thepage}
\renewcommand{\headrulewidth}{0.4pt}
\renewcommand{\footrulewidth}{0pt}

%% ============================================
%% Bibliography
%% ============================================
\usepackage[
  backend=biber,
  style=ieee,
  sorting=nyt,
  maxbibnames=99
]{biblatex}
\addbibresource{bibliography/references.bib}

% URL breaking settings for long URLs in bibliography
\setcounter{biburlnumpenalty}{100}
\setcounter{biburlucpenalty}{100}
\setcounter{biburllcpenalty}{100}

%% ============================================
%% Hyperlinks
%% ============================================
\usepackage{hyperref}
\hypersetup{
  colorlinks=true,
  linkcolor=chaptercolor,
  citecolor=green!50!black,
  urlcolor=blue!70!black,
  pdftitle={כלי בינה מלאכותית בעסקים},
  pdfauthor={דר' יורם סגל ופרופסור ערן שריף}
}

%% ============================================
%% Cross References
%% ============================================
\usepackage{cleveref}

%% ============================================
%% Custom Commands
%% ============================================
% Hebrew/English shortcuts - defined in class file with proper BiDi direction
% \he{} uses \RL{\texthebrew{}} for RTL in LTR context
% \en{} uses \LR{\textenglish{}} for LTR in RTL context

% Technical terms
\newcommand{\term}[1]{\textbf{\textenglish{#1}}}
\newcommand{\heterm}[2]{\textbf{#1} (\textenglish{#2})}

% Code inline
\newcommand{\code}[1]{\texttt{\textenglish{#1}}}

% Math in Hebrew context
%% ============================================
%% Theorem Environments
%% ============================================
\theoremstyle{definition}
\newtheorem{definition}{הגדרה}[chapter]
\newtheorem{example}{דוגמה}[chapter]
\newtheorem{exercise}{תרגיל}[chapter]

\theoremstyle{plain}
\newtheorem{theorem}{משפט}[chapter]
\newtheorem{lemma}{למה}[chapter]

\theoremstyle{remark}
\newtheorem{remark}{הערה}[chapter]
\newtheorem{note}{הערה}[chapter]

%% ============================================
%% Subfiles Support
%% ============================================
\usepackage{subfiles}

%% ============================================
%% Miscellaneous
%% ============================================
\usepackage{enumitem}
\usepackage{float}
\usepackage{caption}
\usepackage{subcaption}

% Set Hebrew list labels
\setlist[itemize]{label=\textbullet}
\setlist[enumerate]{label=\arabic*.}

% Float placement
\renewcommand{\floatpagefraction}{0.8}
\renewcommand{\topfraction}{0.9}
\renewcommand{\bottomfraction}{0.9}

%% ============================================
%% Hebrew RTL Table Support (from CLS v5.10)
%% ============================================
\usepackage{colortbl}  % For \rowcolor

% Hebrew table environment with RTL orientation
\newenvironment{hebrewtable}[1][htbp]{%
  \begin{table}[#1]%
  \begingroup%
  \renewcommand{\arraystretch}{1.2}%
  \captionsetup{justification=centering,singlelinecheck=false}%
  \centering%
}{%
  \endgroup%
  \end{table}%
}

% RTL tabular environment - columns in REVERSE order (right-to-left)
% Use m{width} column type for vertical centering
\newenvironment{rtltabular}[1]{%
  \begin{tabular}{#1}%
}{%
  \end{tabular}%
}

% \hebcell{} - Hebrew RTL table cell with mixed language support
\newcommand{\hebcell}[1]{%
  \bgroup%
  \textdir TRT\pardir TRT%
  \everypar{\textdir TRT\pardir TRT}%
  \selectlanguage{hebrew}%
  \rightskip=0pt\leftskip=0pt plus 1fil\relax%
  \mbox{}\par\vspace{+0.2\baselineskip}\vspace{0.5ex}%
  \ignorespaces#1\unskip%
  \vspace*{0.5ex}%
  \egroup%
}

% \encell{} - English LTR table cell with vertical padding
\newcommand{\encell}[1]{%
  \bgroup%
  \mbox{}\par\vspace{+0.2\baselineskip}\vspace{0.5ex}%
  #1%
  \vspace*{0.5ex}%
  \egroup%
}

% \hebheader{} - Hebrew RTL table header cell
\newcommand{\hebheader}[1]{%
  \bgroup%
  \textdir TRT\pardir TRT%
  \everypar{\textdir TRT\pardir TRT}%
  \selectlanguage{hebrew}%
  \rightskip=0pt\leftskip=0pt plus 1fil\relax%
  \vspace*{0.5ex}%
  \ignorespaces#1\unskip%
  \vspace*{0.5ex}%
  \egroup%
}

% \enheader{} - English LTR table header cell
\newcommand{\enheader}[1]{%
  \bgroup%
  \vspace*{0.5ex}%
  #1%
  \vspace*{0.5ex}%
  \egroup%
}


\begin{document}

%% ============================================
%% Front Matter
%% ============================================
\frontmatter

%% Title Page
\begin{titlepage}
\begin{center}

\vspace*{2cm}

{\Huge\bfseries\color{chaptercolor} כלי בינה מלאכותית בעסקים}

\vspace{0.5cm}

{\Large\textenglish{AI Tools in Business}}

\vspace{0.3cm}

{\large\textenglish{(Agentic AI Engineering)}}

\vspace{2cm}

{\LARGE מאת}

\vspace{0.5cm}

{\Large\bfseries דר' יורם סגל}

\vspace{0.3cm}

{\Large ו}

\vspace{0.3cm}

{\Large\bfseries פרופסור ערן שריף}

\vspace{3cm}

\begin{tikzpicture}[scale=0.8]
  % AI Brain Icon
  \draw[thick, chaptercolor, fill=examplecolor] (0,0) circle (2cm);
  \foreach \i in {0,45,...,315} {
    \draw[thick, sectioncolor] (0,0) -- (\i:1.8);
  }
  \draw[thick, chaptercolor, fill=white] (0,0) circle (0.8cm);
  \node at (0,0) {\Large\textenglish{AI}};

  % Connection nodes
  \foreach \i in {0,60,...,300} {
    \draw[thick, sectioncolor, fill=formulacolor] (\i:2.5) circle (0.3cm);
    \draw[thick, sectioncolor] (\i:2) -- (\i:2.2);
  }
\end{tikzpicture}

\vspace{2cm}

{\large ספר לימוד לתואר שני במנהל עסקים}

\vspace{0.5cm}

{\normalsize מהדורה ראשונה -- 2025}

\vfill

{\small כל הזכויות שמורות לדר' יורם סגל ופרופסור ערן שריף}

\end{center}
\end{titlepage}

%% Copyright Page
\newpage
\thispagestyle{empty}
\vspace*{\fill}
\begin{center}
{\small
כל הזכויות שמורות © 2025

דר' יורם סגל ופרופסור ערן שריף

\vspace{1cm}

אין לשכפל, להעתיק, לצלם, להקליט, לתרגם, לאחסן במאגר מידע,
לשדר או לקלוט בכל דרך או בכל אמצעי אלקטרוני, אופטי או מכני
או אחר -- כל חלק שהוא מהחומר שבספר זה.

שימוש מסחרי מכל סוג שהוא בחומר הכלול בספר זה אסור בהחלט,
אלא ברשות מפורשת בכתב מהמחברים.

\vspace{1cm}

\textenglish{All rights reserved © 2025}

\textenglish{Dr. Yoram Segal \& Prof. Eran Sheriff}
}
\end{center}
\vspace*{\fill}

%% Table of Contents
\tableofcontents

%% List of Figures
\listoffigures

%% List of Tables
\listoftables

%% Preface
\chapter*{הקדמה}
\addcontentsline{toc}{chapter}{הקדמה}

ברוכים הבאים לספר "כלי בינה מלאכותית בעסקים".

אנו חיים בעידן של מהפכה טכנולוגית חסרת תקדים. מודלי שפה גדולים (LLMs) ובינה מלאכותית יוצרת משנים את האופן שבו עסקים פועלים, מקבלים החלטות ומתקשרים עם לקוחותיהם. כמנהלים בעידן הזה, ההבנה של הכלים הללו אינה עוד מותרות -- היא הכרח.

ספר זה נכתב עבור מנהלים ברמת MBA שרוצים להבין את עולם ה-AI העסקי לעומק, אך ללא הצורך להפוך למהנדסי תוכנה. המטרה שלנו היא לספק לכם את הידע הדרוש לקבלת החלטות מושכלות, לתכנון אסטרטגי של הטמעת AI בארגון, ולהובלת פרויקטים בתחום.

הספר מחולק ל-13 פרקים, פרק לכל שבוע בסמסטר. כל פרק בנוי על קודמיו, ויחד הם יוצרים תמונה שלמה של האקוסיסטם של AI בעסקים -- מהיסודות התיאורטיים ועד ליישום מעשי מקצה לקצה.

בכל פרק תמצאו:
\begin{itemize}
  \item הסברים תיאורטיים בסגנון נגיש וסיפורי
  \item נוסחאות מנהליות לחישוב ROI, עלויות והחזר השקעה
  \item תרשימים וגרפים להמחשה ויזואלית
  \item דוגמאות מעשיות מעולם העסקים
  \item תרגילים תיאורטיים ומעשיים עם פתרונות
  \item קוד Python להתנסות מעשית
\end{itemize}

אנו מאחלים לכם למידה פורייה ומהנה.

\vspace{1cm}

\begin{flushright}
דר' יורם סגל ופרופסור ערן שריף

2025
\end{flushright}

%% ============================================
%% Main Matter
%% ============================================
\mainmatter

% Chapter 1: Introduction to LLMs
\subfile{chapters/chapter-01}

% Chapter 2: AI Ecosystem
\subfile{chapters/chapter-02}

% Chapter 3: REST APIs and JSON
\subfile{chapters/chapter-03}

% Chapter 4: MCP Protocol
\subfile{chapters/chapter-04}

% Chapter 5: Autonomous Agents
\subfile{chapters/chapter-05}

% Chapter 6: A2A Protocol
\subfile{chapters/chapter-06}

% Chapter 7: RAG Systems
\subfile{chapters/chapter-07}

% Chapter 8: Prompt Engineering
\subfile{chapters/chapter-08}

% Chapter 9: Deployment
\subfile{chapters/chapter-09}

% Chapter 10: Strategic Considerations
\subfile{chapters/chapter-10}

% Chapter 11: Interfaces and UX
\subfile{chapters/chapter-11}

% Chapter 12: Ethics, Regulation and Security
\subfile{chapters/chapter-12}

% Chapter 13: From Project to Product
\subfile{chapters/chapter-13}

%% ============================================
%% Back Matter
%% ============================================
\backmatter

%% Bibliography
\printbibliography[heading=bibintoc,title={רשימת מקורות}]

%% Index (optional)
% \printindex

%% Appendices
\appendix

\chapter{מילון מונחים}

\begin{longtable}{p{4cm}p{4cm}p{6cm}}
\toprule
\textbf{מונח באנגלית} & \textbf{תרגום לעברית} & \textbf{הסבר} \\
\midrule
\endhead

\textenglish{LLM} & מודל שפה גדול & מודל בינה מלאכותית המאומן על כמויות עצומות של טקסט \\
\textenglish{API} & ממשק תכנות יישומים & דרך סטנדרטית לתקשורת בין מערכות \\
\textenglish{REST} & העברת מצב ייצוגי & ארכיטקטורת תקשורת נפוצה ל-API \\
\textenglish{JSON} & סימון אובייקט JavaScript & פורמט להעברת נתונים \\
\textenglish{MCP} & פרוטוקול הקשר מודל & פרוטוקול להעברת הקשר למודלי AI \\
\textenglish{Agent} & סוכן & תוכנה אוטונומית המבצעת משימות \\
\textenglish{A2A} & סוכן-לסוכן & תקשורת בין סוכנים אוטונומיים \\
\textenglish{RAG} & יצירה מועשרת באחזור & טכניקה לשילוב ידע עדכני ב-LLM \\
\textenglish{Prompt} & פרומפט/הנחיה & הטקסט שנשלח ל-LLM לעיבוד \\
\textenglish{Token} & טוקן & יחידת עיבוד בסיסית של LLM \\
\textenglish{Embedding} & שיבוץ/הטמעה & ייצוג וקטורי של טקסט \\
\textenglish{Vector Database} & בסיס נתונים וקטורי & מאגר לאחסון וחיפוש וקטורים \\
\textenglish{Fine-tuning} & כוונון עדין & התאמת מודל קיים לתחום ספציפי \\
\textenglish{Hallucination} & הזיה & יצירת מידע שגוי על ידי LLM \\
\textenglish{Context Window} & חלון הקשר & כמות הטקסט שהמודל יכול לעבד בבת אחת \\
\bottomrule
\end{longtable}

\chapter{קוד Python מלא}

פרק זה מכיל את כל קטעי הקוד המופיעים בספר במלואם, מוכנים להעתקה והרצה.

% Code snippets will be included here

\end{document}
