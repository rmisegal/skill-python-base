% chapter-01-standalone.tex
% Using hebrew-academic-template.cls v5.10.0
% CLS provides: fontspec, polyglossia, amsmath, amssymb, graphicx, xcolor,
%   tikz-cd, booktabs, tabularx, longtable, array, hyperref, fancyhdr,
%   enumitem, float, caption, titlesec, tcolorbox, biblatex, etc.

\documentclass{../hebrew-academic-template}

%% ============================================
%% Additional Packages (not in CLS)
%% ============================================
\usepackage{luatexbase}
\usepackage{amsthm}
\usepackage{mathtools}
\usepackage{multirow}
\usepackage{subcaption}
\usepackage{listings}

%% ============================================
%% TikZ and PGFPlots Extensions
%% ============================================
\usepackage{tikz}
\usepackage{pgfplots}
\pgfplotsset{compat=1.18}
\usetikzlibrary{shapes,arrows,positioning,calc,fit,backgrounds,decorations.pathreplacing}

%% ============================================
%% Project-Specific Colors
%% ============================================
\definecolor{chaptercolor}{RGB}{0,51,102}
\definecolor{sectioncolor}{RGB}{0,76,153}
\definecolor{examplecolor}{RGB}{230,242,255}
\definecolor{exercisecolor}{RGB}{255,248,220}
\definecolor{codebackground}{RGB}{245,245,245}
\definecolor{formulacolor}{RGB}{240,255,240}

%% ============================================
%% Code Listings Style
%% ============================================
\lstset{
  basicstyle=\ttfamily\small,
  backgroundcolor=\color{codebackground},
  frame=single,
  breaklines=true,
  numbers=left,
  numberstyle=\tiny\color{gray},
  keywordstyle=\color{blue}\bfseries,
  commentstyle=\color{green!50!black},
  stringstyle=\color{red!60!black},
  showstringspaces=false,
  tabsize=4
}

% Python style
\lstdefinestyle{python}{
  language=Python,
  morekeywords={self,True,False,None,as,with,async,await}
}

% JSON style
\lstdefinestyle{json}{
  basicstyle=\ttfamily\small,
  stringstyle=\color{red!60!black},
  morestring=[b]",
  literate=
    *{:}{{{\color{blue}:}}}{1}
    {,}{{{\color{blue},}}}{1}
    {\{}{{{\color{blue}\{}}}{1}
    {\}}{{{\color{blue}\}}}}{1}
    {[}{{{\color{blue}[}}}{1}
    {]}{{{\color{blue}]}}}{1}
}

%% ============================================
%% Project-Specific Boxes (BiDi-safe tcolorbox environments)
%% Pattern: qa-BiDi-fix-tcolorbox - wrapper for RTL background overflow
%% ============================================
\tcbuselibrary{skins,breakable,theorems}

% ============== INTERNAL BOXES (with @inner suffix) ==============

% Example Box - Internal
\newtcolorbox{examplebox@inner}[1][]{
  enhanced,
  breakable,
  colback=examplecolor,
  colframe=sectioncolor,
  fonttitle=\bfseries,
  title={\texthebrew{#1}},
  halign title=flush right,
  arc=3mm,
  boxrule=1pt
}

% Exercise Box - Internal
\newtcolorbox{exercisebox@inner}[1][]{
  enhanced,
  breakable,
  colback=exercisecolor,
  colframe=orange!70!black,
  fonttitle=\bfseries,
  title={\texthebrew{#1}},
  halign title=flush right,
  arc=3mm,
  boxrule=1pt
}

% Formula Box - Internal
\newtcolorbox{formulabox@inner}[1][]{
  enhanced,
  breakable,
  colback=formulacolor,
  colframe=green!50!black,
  fonttitle=\bfseries,
  title={\texthebrew{#1}},
  halign title=flush right,
  arc=2mm,
  boxrule=0.5pt
}

% Code Box - Internal
\newtcolorbox{codebox@inner}[1][]{
  enhanced,
  breakable,
  colback=codebackground,
  colframe=gray!50,
  fonttitle=\bfseries\ttfamily,
  title={\texthebrew{#1}},
  halign title=flush right,
  arc=2mm,
  boxrule=0.5pt
}

% Note Box - Internal
\newtcolorbox{notebox@inner}[1][]{
  enhanced,
  breakable,
  colback=yellow!10,
  colframe=yellow!50!black,
  fonttitle=\bfseries,
  title={\texthebrew{#1}},
  halign title=flush right,
  arc=2mm,
  boxrule=1pt
}

% ============== BiDi-SAFE WRAPPER ENVIRONMENTS ==============
% Force LTR for box drawing, RTL for content inside

\newenvironment{examplebox}[1][]
  {\begin{english}\begin{examplebox@inner}[#1]\selectlanguage{hebrew}}
  {\end{examplebox@inner}\end{english}}

\newenvironment{exercisebox}[1][]
  {\begin{english}\begin{exercisebox@inner}[#1]\selectlanguage{hebrew}}
  {\end{exercisebox@inner}\end{english}}

\newenvironment{formulabox}[1][]
  {\begin{english}\begin{formulabox@inner}[#1]\selectlanguage{hebrew}}
  {\end{formulabox@inner}\end{english}}

\newenvironment{codebox}[1][]
  {\begin{english}\begin{codebox@inner}[#1]\selectlanguage{hebrew}}
  {\end{codebox@inner}\end{english}}

\newenvironment{notebox}[1][]
  {\begin{english}\begin{notebox@inner}[#1]\selectlanguage{hebrew}}
  {\end{notebox@inner}\end{english}}

%% ============================================
%% Chapter/Section Styling Override
%% ============================================
% Note: CLS uses article class which has no \chapter - use \hebrewchapter instead
% \titleformat{\chapter} removed since article class doesn't define \chapter

\titleformat{\section}
  {\normalfont\Large\bfseries\color{sectioncolor}}
  {\textenglish{\thesection}}{1em}{}

\titleformat{\subsection}
  {\normalfont\large\bfseries}
  {\textenglish{\thesubsection}}{1em}{}

%% ============================================
%% Custom Commands (project-specific)
%% ============================================
% Hebrew/English shortcuts (CLS provides \en, \heb already)
\providecommand{\he}[1]{\texthebrew{#1}}

% Technical terms
\newcommand{\term}[1]{\textbf{\textenglish{#1}}}
\newcommand{\heterm}[2]{\textbf{#1} (\textenglish{#2})}

%% ============================================
%% Theorem Environments
%% ============================================
\theoremstyle{definition}
\newtheorem{definition}{הגדרה}[section]
\newtheorem{example}{דוגמה}[section]
\newtheorem{exercise}{תרגיל}[section]

\theoremstyle{plain}
\newtheorem{theorem}{משפט}[section]
\newtheorem{lemma}{למה}[section]

\theoremstyle{remark}
\newtheorem{remark}{הערה}[section]
\newtheorem{note}{הערה}[section]

%% ============================================
%% Float Configuration
%% ============================================
\renewcommand{\floatpagefraction}{0.7}
\renewcommand{\topfraction}{0.9}
\renewcommand{\bottomfraction}{0.9}
\renewcommand{\textfraction}{0.1}

%% Allow more floats per page
\setcounter{topnumber}{3}
\setcounter{bottomnumber}{3}
\setcounter{totalnumber}{5}

%% ============================================
%% Page Layout - Fix underfull vbox warnings
%% (qa-typeset-fix-vbox: Option B - Global)
%% ============================================
\raggedbottom  % Allow uneven page bottoms instead of stretching

%% ============================================
%% Bibliography
%% ============================================
\addbibresource{../bibliography/references.bib}

%% ============================================
%% Book Metadata for Standalone
%% ============================================
\title{כלי בינה מלאכותית בעסקים}
\author{דר' יורם סגל ופרופסור ערן שריף}
\date{2025}

% Latin environment for LTR code blocks
\newenvironment{latin}{\begin{english}}{\end{english}}

% Abstract environment for book class (using renewenvironment since amsthm defines abstract)
\renewenvironment{abstract}{%
  \begin{center}%
    \bfseries תקציר%
  \end{center}%
  \begin{quotation}%
}{%
  \end{quotation}%
}

\begin{document}
% Chapter 1: המהפכה השקטה - מבוא למודלי שפה גדולים בעולם העסקי
% Authors: דר' יורם סגל ופרופסור ערן שריף

\hebrewchapter{המהפכה השקטה - מבוא למודלי שפה גדולים בעולם העסקי}
\label{chap:llm_intro}

\begin{abstract}
\begin{hebrew}
בעידן שבו טכנולוגיה משנה את עולם העסקים בקצב מסחרר, קמה מהפכה חדשה - שקטה אך עמוקה. מודלי שפה גדולים (\textenglish{Large Language Models, LLMs}) משנים את האופן שבו ארגונים מתקשרים, מנתחים מידע ומקבלים החלטות. פרק זה מציג את היסודות המנהליים להבנת הטכנולוגיה, פוטנציאלה ומגבלותיה, ומספק כלים מעשיים להערכת התועלת העסקית שלה.
\end{hebrew}
\end{abstract}

\section*{מטרות הלמידה}
\begin{hebrew}
בסיום פרק זה, הקורא יוכל:
\begin{itemize}
    \item להבין את המהות והפוטנציאל של מודלי שפה גדולים עבור ארגונים
    \item לזהות את שני התפקידים המרכזיים: אינטראקציה בשפה טבעית ועיבוד לוגיקה מורכבת
    \item להכיר את נקודות החוזק והחולשה של \textenglish{LLMs} לצורך קבלת החלטות מושכלות
    \item לחשב ולהעריך את התשואה על ההשקעה (\textenglish{ROI}) ביישום כלי \textenglish{AI}
    \item לזהות תרחישי שימוש מתאימים ובלתי מתאימים למודלים אלו
\end{itemize}
\end{hebrew}

\section{פרולוג: בוקר רגיל בעולם חדש}
\label{sec:prologue}

\begin{hebrew}
שבע בבוקר. שרה, מנהלת שיווק בחברת \textenglish{SaaS} בינונית, מתיישבת מול המחשב עם כוס קפה. לפניה משימה מוכרת: כתיבת חמישה פוסטים לרשתות חברתיות לקמפיין החדש. בעבר, זה היה לוקח לה שעתיים לפחות. היום, היא פותחת את \textenglish{ChatGPT}, מקלידה הנחיה קצרה עם ההקשר והסגנון המבוקש, וכעבור דקה וחצי - חמשת הפוסטים מוכנים. היא משקיעה עוד עשר דקות בעריכה ועיצוב, ועוברת למשימה הבאה.

באותו זמן, דן, סמנכ"ל הכספים, מעלה לממשק \textenglish{Claude} דוח כספי בן 50 עמודים ומבקש סיכום של המגמות העיקריות ונקודות החריגה. כעבור שתי דקות, הוא מקבל ניתוח מובנה שבעבר היה דורש מבקר פיננסי שעה שלמה. הוא לא מסתפק בכך - הוא ממשיך לשאול שאלות המשך, ו-\textenglish{Claude} עונה בהקשר מלא, כאילו הוא עמית שקרא את הדוח בעצמו.

בקומה השלישית, רונית ממשאבי אנוש מעבירה ראיון ראשוני עם מועמד. היא לא לבד - לידה פועל סוכן \textenglish{AI} שמקליט, מתמלל ומנתח בזמן אמת את תשובות המועמד מול פרופיל התפקיד. כשהשיחה מסתיימת, רונית כבר רואה דוח מסכם עם המלצה ראשונית.

זהו לא מדע בדיוני. זה לא עתיד רחוק. זה היום, עכשיו, בעשרות אלפי ארגונים ברחבי העולם.

אבל מה באמת קורה כאן? מה הופך את הטכנולוגיה הזו לשונה מכל אוטומציה שראינו עד כה? ואיך מנהלים אמורים להבין, להעריך ולהטמיע אותה בארגון שלהם?
\end{hebrew}

\section{מהם מודלי שפה גדולים? הסבר אינטואיטיבי}
\label{sec:what_are_llms}

\subsection{המטאפורה: מכונת השלמת דפוסים}
\label{subsec:completion_machine}

\begin{hebrew}
דמיינו לרגע ילד שגדל בסביבה שבה הוא שומע מיליוני שיחות, קורא מיליארדי משפטים, וחשוף לכמעט כל נושא אנושי אפשרי - היסטוריה, מדע, ספרות, עסקים, פילוסופיה. הילד הזה לא מבין בהכרח את העולם כמו שאנחנו מבינים אותו, אבל הוא מפתח יכולת מדהימה לזהות דפוסים: איך משפטים בנויים, איך רעיונות מתקשרים, איך בעיות נפתרות, איך אנשים מתקשרים בהקשרים שונים.

כשאתם שואלים את הילד הזה שאלה, הוא לא מחפש תשובה במאגר מידע. במקום זאת, הוא משתמש בכל הדפוסים שהוא למד כדי \textbf{להשלים} את המשפט הכי הגיוני, הכי סביר, הכי מתאים להקשר. אם שאלתם על אסטרטגיית שיווק, הוא יזכור מיליוני שיחות על שיווק שהוא "שמע", וישלים את התשובה בצורה שמשקפת את הדפוסים האלה.

זו, בקצרה, המהות של \textenglish{Large Language Model (LLM)}.

\textenglish{LLM} הוא מודל מתמטי ענק שאומן על כמויות אדירות של טקסט - ספרים, מאמרים, אתרי אינטרנט, קוד תוכנה, ועוד. בתהליך האימון, המודל למד דפוסים סטטיסטיים מורכבים:
\begin{itemize}
    \item איזה מילים מופיעות לצד אילו מילים
    \item איך משפטים בנויים בהקשרים שונים
    \item איך רעיונות מתקשרים זה לזה
    \item איך בעיות נפתרות בתחומים שונים
\end{itemize}

כשאתם כותבים \textenglish{prompt} (הנחיה) ל-\textenglish{LLM}, המודל "רואה" את הטקסט שלכם ומשתמש בכל הדפוסים שהוא למד כדי לחזות את ההמשך הכי סביר. הוא עושה זאת מילה אחר מילה, תוך התחשבות בכל ההקשר שלפניו.

\textbf{נקודה קריטית למנהלים:} \textenglish{LLM} לא "יודע" דברים במובן האנושי. הוא לא מחפש מידע במסד נתונים. הוא יוצר טקסט חדש על בסיס דפוסים סטטיסטיים. זו גם החוזקה (יצירתיות, גמישות) וגם החולשה (אפשרות להזיות) שלו.
\end{hebrew}

\subsection{מתחת למכסה המנוע: ארכיטקטורה בסיסית}
\label{subsec:basic_architecture}

\begin{hebrew}
בלי להיכנס לפרטים טכניים עמוקים מדי, חשוב להבין את המבנה הבסיסי של \textenglish{LLM}:
\end{hebrew}

\begin{figure}[htbp]
\centering
\begin{english}
\begin{tikzpicture}[
    node distance=2.5cm,
    block/.style={rectangle, draw, fill=blue!20, text width=6em, text centered, rounded corners, minimum height=3em},
    arrow/.style={->, >=stealth, thick}
]
    % Nodes - English only for proper TikZ rendering
    \node[block] (input) {\textbf{Input}\\Textual Input};
    \node[block, right of=input] (tokenize) {\textbf{Tokenization}\\Token Splitting};
    \node[block, right of=tokenize] (embedding) {\textbf{Embedding}\\Numeric Vectors};
    \node[block, below of=embedding] (transformer) {\textbf{Transformer}\\Deep Processing};
    \node[block, left of=transformer] (predict) {\textbf{Prediction}\\Next Token};
    \node[block, left of=predict] (output) {\textbf{Output}\\Textual Output};

    % Arrows
    \draw[arrow] (input) -- (tokenize);
    \draw[arrow] (tokenize) -- (embedding);
    \draw[arrow] (embedding) -- (transformer);
    \draw[arrow] (transformer) -- (predict);
    \draw[arrow] (predict) -- (output);
\end{tikzpicture}
\end{english}
\caption{ארכיטקטורה בסיסית של \textenglish{LLM} - מקלט לפלט}
\label{fig:llm_architecture}
\end{figure}

\begin{hebrew}
\begin{enumerate}
    \item \textbf{\textenglish{Input} (קלט):} המשתמש מזין טקסט - שאלה, בקשה, או הקשר.

    \item \textbf{\textenglish{Tokenization} (טוקניזציה):} הטקסט מפורק ל"טוקנים" - יחידות בסיסיות שהמודל מבין. טוקן יכול להיות מילה, חלק ממילה, או סימן פיסוק. למשל, המשפט "שלום עולם" עשוי להיות \textenglish{2-3} טוקנים.

    \item \textbf{\textenglish{Embedding} (הטמעה):} כל טוקן הופך לייצוג מתמטי - וקטור של מספרים שמייצג את משמעותו ביחס לטוקנים אחרים.

    \item \textbf{\textenglish{Transformer} (טרנספורמר):} זוהי הלב של המודל \cite{vaswani2017attention}. רשת נוירונים עמוקה שמנתחת את היחסים בין הטוקנים, מבינה הקשר, ומחלצת משמעות. כאן קורה "הקסם" - המודל "מבין" מה נשאל וכיצד לענות.

    \item \textbf{\textenglish{Prediction} (חיזוי):} המודל מחזה את הטוקן הבא הכי סביר, מוסיף אותו לרצף, וחוזר על התהליך עד שהתשובה שלמה.

    \item \textbf{\textenglish{Output} (פלט):} הטוקנים הופכים בחזרה לטקסט קריא שהמשתמש רואה.
\end{enumerate}

\textbf{למה זה חשוב למנהלים?}
\begin{itemize}
    \item \textbf{טוקנים = עלות:} רוב שירותי ה-\textenglish{API} גובים תשלום לפי מספר הטוקנים שנשלחים (קלט) ומתקבלים (פלט). הבנת טוקנים חיונית לתכנון תקציב.
    \item \textbf{הקשר מוגבל:} לכל מודל יש "חלון הקשר" (\textenglish{context window}) - מספר מקסימלי של טוקנים שהוא יכול לעבד בבת אחת. למשל, \textenglish{GPT-4 Turbo} תומך ב-\textenglish{128,000} טוקנים (~\textenglish{96,000} מילים), בעוד \textenglish{GPT-3.5} תומך רק ב-\textenglish{16,000}.
    \item \textbf{מהירות תלויה בגודל:} מודלים גדולים יותר (יותר פרמטרים) בדרך כלל מדויקים יותר, אך איטיים ויקרים יותר.
\end{itemize}
\end{hebrew}

\section{שני הכוחות העל של \textenglish{LLM}}
\label{sec:two_superpowers}

\subsection{כוח ראשון: תקשורת טבעית עם מכונה}
\label{subsec:natural_communication}

\begin{hebrew}
במשך עשרות שנים, האינטראקציה שלנו עם מחשבים הייתה מוגבלת. רצינו שהמחשב יעשה משהו? היינו צריכים ללמוד את שפתו: לחצנים, תפריטים, שורות פקודה, שפות תכנות. המחשב לא הבין אותנו - אנחנו היינו צריכים להתאים את עצמנו אליו.

\textenglish{LLMs} הופכים את המשוואה. לראשונה בתולדות המחשוב, אנחנו יכולים לתקשר עם מכונה \textbf{בשפה שלנו}, בדיוק כפי שהיינו מדברים עם עמית.

\textbf{דוגמה מעולם העסקים:}

\textbf{בעבר} (אינטראקציה מסורתית עם תוכנה):
\begin{enumerate}
    \item פתח תוכנת \textenglish{Excel}
    \item בחר טווח תאים
    \item לחץ על "נתונים" $\rightarrow$ "סינון" $\rightarrow$ "סינון מתקדם"
    \item הגדר קריטריונים מורכבים
    \item בחר "עותק למיקום אחר"
    \item בחר תא יעד
    \item לחץ "אישור"
\end{enumerate}

\textbf{היום} (אינטראקציה עם \textenglish{LLM}):
\begin{quote}
"תסנן את הטבלה הזו ותראה לי רק לקוחות מאזור המרכז שרכשו מעל \textenglish{10,000} ש"ח ברבעון האחרון"
\end{quote}

המודל מבין את הכוונה, מזהה את הנתונים הרלוונטיים, ומבצע את הפעולה - או אפילו כותב לכם את הנוסחה המתאימה.

\textbf{ההשלכות העסקיות:}
\begin{itemize}
    \item \textbf{הפחתת מחסום הכניסה:} עובדים לא צריכים להיות מומחי תוכנה כדי לבצע משימות מורכבות.
    \item \textbf{מהירות:} מה שלוקח \textenglish{10} דקות בממשק מסורתי, לוקח \textenglish{10} שניות בשפה טבעית.
    \item \textbf{גמישות:} אפשר לשאול שאלות המשך, לשנות דרישות, לחקור כיוונים שונים - בדיוק כמו בשיחה אנושית.
\end{itemize}
\end{hebrew}

\subsection{כוח שני: עיבוד לוגיקה מורכבת}
\label{subsec:complex_logic}

\begin{hebrew}
אבל \textenglish{LLMs} הם הרבה יותר מסתם ממשק נוח. הם מסוגלים לבצע \textbf{חשיבה מורכבת} על נתונים ורעיונות.

בואו נבחן כמה יכולות מרכזיות:

\subsubsection{סיכום והפקת תובנות}

ניתן להזין ל-\textenglish{LLM} דוח של \textenglish{100} עמודים ולבקש:
\begin{itemize}
    \item "סכם את הנקודות העיקריות ב-\textenglish{5} משפטים"
    \item "מה הטרנדים המרכזיים שמופיעים פה?"
    \item "איזה נושאים חוזרים על עצמם?"
\end{itemize}

המודל קורא, מזהה דפוסים, ומפיק תובנות - עבודה שעד לפני כמה שנים דרשה אנליסט אנושי.

\subsubsection{השוואה וניתוח}

"השווה בין שלוש הצעות המחיר האלה מבחינת עלות, זמן אספקה ושירות, והמלץ על הספק המתאים ביותר לארגון שלנו."

המודל לא רק משווה - הוא \textbf{מנמק} את ההמלצה שלו על בסיס הקריטריונים שהגדרתם.

\subsubsection{פתרון בעיות רב-שלבי}

\textenglish{LLMs} מודרניים יכולים לפתור בעיות שדורשות מספר שלבי חשיבה:
\begin{enumerate}
    \item הבנת הבעיה
    \item פירוק לתת-בעיות
    \item פתרון כל תת-בעיה
    \item שילוב התוצאות לפתרון כולל
\end{enumerate}

\textbf{דוגמה:}
\begin{quote}
"יש לנו \textenglish{5} נציגי מכירות, \textenglish{120} לידים חדשים החודש, וכל נציג יכול לטפל בממוצע ב-\textenglish{25} לידים בחודש. איך כדאי לחלק את הלידים בהתחשב בכך שנציג א' מתמחה בלקוחות ארגוניים, ב' ו-ג' בעסקים קטנים, ד' בסטארטאפים, וה' חדש ועדיין מתאמן?"
\end{quote}

המודל יבנה תוכנית חלוקה מפורטת, יסביר את ההיגיון מאחוריה, ואפילו יתריע על בעיות פוטנציאליות.

\subsubsection{תכנות וכתיבת קוד}

\textenglish{LLMs} כמו \textenglish{GPT-4} \cite{openai2023gpt4} ו-\textenglish{Claude} \cite{anthropic2024claude} מסוגלים לכתוב קוד תוכנה באיכות גבוהה. מנהל בלי רקע תכנותי יכול לבקש:

\begin{quote}
"תכתוב סקריפט \textenglish{Python} שקורא קובץ \textenglish{Excel}, מחשב את סכום המכירות לכל מוצר, ויוצר גרף עמודות"
\end{quote}

והמודל יכתוב קוד מלא, מתועד, ומוכן להרצה.

\textbf{השלכה עסקית:} זה מוריד את המחסום לאוטומציה. משימות שבעבר דרשו מתכנת, היום יכולות להתבצע על ידי כל מנהל עם רעיון ברור.
\end{hebrew}

\subsection{המשמעות המשולבת: עובד דיגיטלי}
\label{subsec:digital_worker}

\begin{hebrew}
כשמשלבים את שני הכוחות האלה - תקשורת טבעית ועיבוד לוגיקה מורכבת - מקבלים משהו חדש לחלוטין בעולם העסקים: \textbf{עובד דיגיטלי} שאפשר להדריך, לשאול שאלות, לתקן, ולשפר - בדיוק כמו עובד אנושי.

זה לא עוד כלי שמבצע משימה אחת קבועה. זה ישות דיגיטלית שיכולה:
\begin{itemize}
    \item להבין הוראות מורכבות
    \item להתאים את עצמה למצבים שונים
    \item לשאול שאלות הבהרה
    \item ללמוד מדוגמאות שאתם נותנים
    \item להציע שיפורים
\end{itemize}

\textbf{זו המהפכה האמיתית של \textenglish{LLMs}.}
\end{hebrew}

\section{נקודות החוזק: מה \textenglish{LLMs} עושים טוב במיוחד}
\label{sec:strengths}

\subsection{יצירתיות והפקת רעיונות}
\label{subsec:creativity}

\begin{hebrew}
\textenglish{LLMs} מצטיינים ביצירת תוכן חדש ומגוון:
\begin{itemize}
    \item \textbf{תוכן שיווקי:} מודעות, פוסטים, דפי נחיתה, מיילים
    \item \textbf{תוכן טכני:} מסמכי דרישות, הצעות מחיר, תיעוד
    \item \textbf{רעיונות:} סיעור מוחות אוטומטי, זוויות חדשות לבעיות קיימות
\end{itemize}

\textbf{דוגמה מעשית:}

מנהלת שיווק בחברת \textenglish{B2B SaaS} צריכה \textenglish{10} כותרות שונות לקמפיין \textenglish{LinkedIn Ads} שמקדם כלי ניהול פרויקטים לצוותי פיתוח.

היא כותבת ל-\textenglish{ChatGPT}:
\begin{quote}
"צור \textenglish{10} כותרות לקמפיין לינקדאין למוצר ניהול פרויקטים לצוותי פיתוח תוכנה. המוצר חוסך \textenglish{20}\% מזמן ההנהלה ומשפר שיתוף פעולה. קהל יעד: \textenglish{VP R\&D} וראשי צוותים. טון: מקצועי אך לא יבש, דגש על תוצאות עסקיות."
\end{quote}

תוך שניות, היא מקבלת \textenglish{10} אפשרויות, כל אחת בזווית שונה - \textenglish{ROI}, פריון, איכות חיים, תחרותיות. היא יכולה לבחור, לשלב, או לבקש עוד וריאציות.

\textbf{זמן שנחסך:} שעה של סיעור מוחות ומחשבה $\rightarrow$ \textenglish{5} דקות.
\end{hebrew}

\subsection{הבנת הקשר ונואנסים}
\label{subsec:context_understanding}

\begin{hebrew}
בניגוד לכלי אוטומציה מסורתיים שפועלים לפי כללים קשיחים, \textenglish{LLMs} מבינים הקשר:
\begin{itemize}
    \item הם "זוכרים" את כל השיחה ומתייחסים אליה
    \item הם מבינים כוונות, לא רק מילים
    \item הם מתאימים את התשובה לטון ולסגנון המבוקש
\end{itemize}

\textbf{דוגמה מתמיכת לקוחות:}

לקוח כותב: "המערכת שוב קרסה לי באמצע פעולה חשובה"

\textenglish{Chatbot} מסורתי (מבוסס כללים) יזהה את המילה "קרסה" ויחזיר:
\begin{quote}
"אנא נסה לאתחל את המערכת ולנקות את הזיכרון הזמני."
\end{quote}

סוכן \textenglish{LLM} מבין את ההקשר הרגשי והעסקי:
\begin{quote}
"אני מבין את התסכול - זה בוודאי מעצבן כשזה קורה בעיצומה של עבודה חשובה. בואו נפתור את זה מהר. האם הבעיה קרתה בפעולה ספציפית? זה יעזור לי לאבחן את הגורם ולוודא שזה לא יקרה שוב."
\end{quote}

ההבדל? אמפתיה, הקשר עסקי, וכוונה לפתרון ארוך טווח - לא רק טיפול תסמיני.
\end{hebrew}

\subsection{גמישות והתאמה}
\label{subsec:flexibility}

\begin{hebrew}
\textenglish{LLMs} לא דורשים תכנות מראש לכל תרחיש. אפשר "ללמד" אותם "בדרך":
\begin{itemize}
    \item \textbf{\textenglish{Few-Shot Learning}} \cite{brown2020language}: תן כמה דוגמאות, והמודל יבין את הדפוס
    \item \textbf{התאמה לסגנון:} "כתוב בסגנון פורמלי/חברי/טכני"
    \item \textbf{שינוי כיוון באמצע:} "לא, תשנה את הגישה ל..."
\end{itemize}

\textbf{דוגמה - קטגוריזציה של פניות:}

חברה מקבלת מאות פניות ביום לתמיכת לקוחות. היא רוצה לקטגר אותן אוטומטית ל-\textenglish{5} קטגוריות.

\textbf{גישה מסורתית:} שכירת מתכנת שיבנה מודל \textenglish{ML} מותאם אישית, יאסוף נתוני אימון, ויקח שבועיים.

\textbf{גישה \textenglish{LLM}:}
\begin{quote}
"קטגר את הפניות הבאות לאחת מחמש הקטגוריות: טכני, חיוב, שאלת מכירה, תלונה, בקשת פיצ'ר. הנה שלוש דוגמאות:

[דוגמה 1...]
[דוגמה 2...]
[דוגמה 3...]

עכשיו, קטגר את הפניות האלה:
[רשימת פניות...]"
\end{quote}

המודל יבין את הדפוס מהדוגמאות ויקטגר נכון 90-95\% מהפניות.

\textbf{זמן יישום:} שבועיים $\rightarrow$ \textenglish{30} דקות.
\end{hebrew}

\subsection{עבודה עם שפות מרובות}
\label{subsec:multilingual}

\begin{hebrew}
\textenglish{LLMs} מודרניים כמו \textenglish{GPT-4}, \textenglish{Claude}, ו-\textenglish{Gemini} מדברים עשרות שפות בצורה שוטפת. זה פותח אפשרויות:
\begin{itemize}
    \item תרגום אוטומטי איכותי (מעבר לתרגום מילה במילה - הבנת הקשר תרבותי)
    \item תמיכת לקוחות רב-לשונית בלי צוות עצום
    \item יצירת תוכן בשפות מרובות באופן מיידי
\end{itemize}

\textbf{דוגמה:} חברה ישראלית שמוכרת לאירופה צריכה לתרגם מסמך טכני מעברית לגרמנית, צרפתית, וספרדית. במקום שלושה מתרגמנים מקצועיים ומספר ימים, \textenglish{Claude} מתרגם את שלושת הגרסאות תוך דקות, עם שמירה על טרמינולוגיה טכנית עקבית.
\end{hebrew}

\section{נקודות החולשה: מה \textenglish{LLMs} לא עושים טוב}
\label{sec:weaknesses}

\subsection{הזיות \textenglish{(Hallucinations)}}
\label{subsec:hallucinations}

\begin{hebrew}
זוהי אולי החולשה הקריטית ביותר של \textenglish{LLMs}: הנטייה "להמציא" מידע \cite{huang2024hallucination}.

זכרו - \textenglish{LLM} הוא מכונת השלמת דפוסים. הוא לא מחפש מידע במסד נתונים; הוא מחזה את המשך הסביר ביותר. לפעמים, אם המידע הנכון לא קיים בזיכרון הסטטיסטי שלו, המודל יייצר עובדות שנשמעות מהימנות - אבל שגויות לחלוטין.

\textbf{דוגמה מסוכנת:}

עורך דין ביקש מ-\textenglish{ChatGPT} לספק תקדימים משפטיים לתמיכה בתביעה. המודל מסר רשימה של שישה תקדימים, כולל שמות תיקים, מספרי תיק, ותאריכים. הכל נראה לגיטימי.

הבעיה? \textbf{אף אחד מהתקדימים לא היה אמיתי.} \textenglish{ChatGPT} המציא אותם, כי הם נשמעו הגיוניים בהקשר.

המשפט הסתיים בסנקציות חמורות על עורך הדין.

\textbf{למה זה קורה?}

\textenglish{LLM} נועד "להישמע" מהימן ורהוט. אין לו מנגנון פנימי שאומר "אני לא יודע". במקום זאת, הוא ממשיך לייצר את המשך הכי סביר, גם אם זה לא מבוסס עובדות.

\textbf{השלכות עסקיות:}
\begin{itemize}
    \item \textbf{אין לסמוך על \textenglish{LLM} לעובדות ללא אימות.} תמיד יש לבדוק מידע קריטי.
    \item \textbf{מתאים ליצירת רעיונות וטיוטות ראשוניות}, פחות מתאים לדוחות עובדתיים סופיים.
    \item \textbf{חובה לשלב בקרה אנושית} במערכות קריטיות.
\end{itemize}
\end{hebrew}

\subsection{חוסר עדכניות \textenglish{(Knowledge Cutoff)}}
\label{subsec:knowledge_cutoff}

\begin{hebrew}
כל \textenglish{LLM} נאמן עד תאריך מסוים - ה"\textenglish{knowledge cutoff}" שלו. למשל:
\begin{itemize}
    \item \textenglish{GPT-4} (גרסה מ-2024): נתונים עד אפריל 2023
    \item \textenglish{Claude 3.5 Sonnet}: נתונים עד אפריל 2024
\end{itemize}

המשמעות: המודל לא יודע כלום על אירועים, מוצרים, טכנולוגיות, או שינויים שקרו אחרי התאריך הזה.

\textbf{דוגמה:}

CFO שואל את \textenglish{GPT-4}:
\begin{quote}
"מה שער הדולר מול השקל היום?"
\end{quote}

התשובה תהיה מבוססת על נתונים ישנים, או לחלופין - הזיה.

\textbf{פתרונות:}
\begin{itemize}
    \item שימוש במודלים עם גישה לאינטרנט (כמו \textenglish{ChatGPT Plus} עם \textenglish{browsing mode})
    \item שילוב \textenglish{RAG (Retrieval-Augmented Generation)} \cite{lewis2020retrieval} - הזרקת מידע עדכני למודל
    \item שימוש ב-\textenglish{APIs} חיצוניים שהסוכן יכול לקרוא להם
\end{itemize}

\textbf{הנפקות עסקיות:}
בתחומים דינמיים (פיננסים, חדשות, נתונים תפעוליים), אין להסתמך על \textenglish{LLM} לבדו. יש לספק לו מידע עדכני או לשלב אותו עם מקורות מידע חיים.
\end{hebrew}

\subsection{עלויות - לא זניח}
\label{subsec:costs}

\begin{hebrew}
שימוש ב-\textenglish{LLMs} דרך \textenglish{API} עולה כסף, והעלות יכולה להפתיע ארגונים שלא תכננו נכון.

מודלים מתומחרים לפי \textbf{טוקנים}:
\end{hebrew}

\begin{table}[htbp]
\centering
\begin{latin}
\begin{tabular}{|l|r|r|}
\hline
\textbf{Model} & \textbf{Input (\$/1M tokens)} & \textbf{Output (\$/1M tokens)} \\
\hline
GPT-4 Turbo & 10.00 & 30.00 \\
GPT-3.5 Turbo & 0.50 & 1.50 \\
Claude 3.5 Sonnet & 3.00 & 15.00 \\
Claude 3 Haiku & 0.25 & 1.25 \\
Gemini 1.5 Pro & 1.25 & 5.00 \\
\hline
\end{tabular}
\end{latin}
\caption{\he{מחירי מודלים נפוצים (נכון למרץ 2024)}}
\label{tab:model_pricing}
\end{table}

\begin{hebrew}
\textbf{דוגמת חישוב:}

נניח שחברה משתמשת ב-\textenglish{GPT-4 Turbo} לתמיכת לקוחות. כל שיחה:
\begin{itemize}
    \item קלט ממוצע: \textenglish{2,000} טוקנים (הקשר + שאלת הלקוח)
    \item פלט ממוצע: \textenglish{800} טוקנים (תשובה)
\end{itemize}

עלות לשיחה:
\begin{equation}
\begin{split}
\text{Cost} &= \left(\frac{2{,}000}{1{,}000{,}000} \times 10\right) + \left(\frac{800}{1{,}000{,}000} \times 30\right) \\
&= 0.02 + 0.024 \\
&= \$0.044
\end{split}
\end{equation}

זה נשמע זניח, נכון? אבל אם יש \textenglish{10,000} שיחות בחודש:
\begin{equation}
\text{Monthly Cost} = 10{,}000 \times 0.044 = \$440
\end{equation}

ואם זה גדל ל-\textenglish{100,000} שיחות בחודש (חברה גדולה):
\begin{equation}
\text{Monthly Cost} = 100{,}000 \times 0.044 = \$4{,}400/\text{month} = \$52{,}800/\text{year}
\end{equation}

\textbf{אסטרטגיות חיסכון:}
\begin{itemize}
    \item שימוש במודלים זולים יותר למשימות פשוטות (\textenglish{GPT-3.5} במקום \textenglish{GPT-4})
    \item אופטימיזציה של \textenglish{prompts} להיות קצרים וממוקדים
    \item \textenglish{Caching} - שמירת תשובות לשאלות נפוצות
    \item שימוש במודלים \textenglish{self-hosted} (למשל \textenglish{Llama 3}) לנפחים גדולים
\end{itemize}
\end{hebrew}

\subsection{חוסר שקיפות \textenglish{(Black Box)}}
\label{subsec:black_box}

\begin{hebrew}
\textenglish{LLMs} הם "קופסה שחורה". כשהם נותנים תשובה, אי אפשר לדעת בדיוק \textbf{למה} הם הגיעו למסקנה הזו. אין "שרשרת הוכחה".

\textbf{בעיה עסקית:}
\begin{itemize}
    \item \textbf{רגולציה:} בתחומים מוסדרים (בנקאות, בריאות), לפעמים נדרש להסביר החלטות. "כי ה-\textenglish{AI} אמר" לא מספיק.
    \item \textbf{אמון:} מנהלים מתקשים לסמוך על מערכת שלא יכולה להסביר את ההיגיון שלה.
    \item \textbf{\textenglish{Bias}:} קשה לזהות ולתקן הטיות כשלא רואים את תהליך החשיבה.
\end{itemize}

\textbf{פתרונות חלקיים:}
\begin{itemize}
    \item שימוש ב-\textenglish{Chain-of-Thought prompting} \cite{wei2022chain} - לבקש מהמודל להסביר את הצעדים שלו
    \item שילוב מערכות \textenglish{explainable AI} משלימות
    \item בקרה אנושית במקרים קריטיים
\end{itemize}
\end{hebrew}

\section{נוסחאות מנהליות להערכת \textenglish{LLMs}}
\label{sec:formulas}

\subsection{מדד \textenglish{ROI} של יישום \textenglish{AI}}
\label{subsec:roi_formula}

\begin{hebrew}
לפני השקעה בכלי \textenglish{AI}, חשוב לחשב את התשואה הצפויה על ההשקעה \cite{brynjolfsson2023generative, almousa2024llmroi}.

\textbf{נוסחת \textenglish{ROI} בסיסית:}
\end{hebrew}

\begin{equation}
\text{ROI} = \frac{(\hebmath{שעות נחסכות} \times \hebmath{עלות שעת עבודה}) - \hebmath{עלות מנוי AI}}{\hebmath{עלות מנוי AI}} \times 100\%
\label{eq:roi_basic}
\end{equation}

\begin{hebrew}
\textbf{דוגמה מעשית:}

צוות תמיכת לקוחות בן \textenglish{5} אנשים משתמש ב-\textenglish{Claude} לטיפול בפניות שגרתיות.

\textbf{נתונים:}
\begin{itemize}
    \item כל נציג מטפל ב-\textenglish{30} פניות ביום
    \item לפני \textenglish{AI}: זמן ממוצע לפנייה = \textenglish{15} דקות
    \item עם \textenglish{AI} (סיוע בכתיבה, חיפוש מידע, תבניות): זמן ממוצע = \textenglish{10} דקות
    \item חיסכון לפנייה: \textenglish{5} דקות
    \item ימי עבודה בחודש: \textenglish{22}
    \item עלות שעת עבודה ממוצעת: \textenglish{100} ש"ח
    \item עלות מנוי \textenglish{Claude Pro} לכל נציג: \textenglish{\$20}/חודש = \textenglish{75} ש"ח
\end{itemize}

\textbf{חישוב:}

שעות נחסכות בחודש:
\begin{equation}
\begin{split}
\hebmath{שעות נחסכות} &= 5 \hebmath{ נציגים} \times 30 \hebmath{ פניות/יום} \times \frac{5}{60} \hebmath{ שעות} \times 22 \hebmath{ ימים} \\
&= 5 \times 30 \times 0.0833 \times 22 \\
&= 275 \hebmath{ שעות}
\end{split}
\end{equation}

ערך החיסכון:
\begin{equation}
\hebmath{ערך} = 275 \times 100 = 27{,}500 \hebmath{ ש"ח}
\end{equation}

עלות מנוי:
\begin{equation}
\hebmath{עלות} = 5 \times 75 = 375 \hebmath{ ש"ח}
\end{equation}

\textenglish{ROI}:
\begin{equation}
\text{ROI} = \frac{27{,}500 - 375}{375} \times 100\% = 7{,}233\%
\end{equation}

\textbf{משמעות:} על כל שקל שהושקע ב-\textenglish{AI}, החברה חוסכת \textenglish{72} שקל. זו תשואה עצומה.

\textbf{נוסחה מורחבת} (כוללת עלויות נוספות):
\end{hebrew}

\begin{equation}
\text{ROI} = \frac{\hebmath{תועלת כוללת} - \hebmath{עלות כוללת}}{\hebmath{עלות כוללת}} \times 100\%
\end{equation}

\begin{hebrew}
כאשר:
\begin{itemize}
    \item \textbf{תועלת כוללת} = חיסכון בשעות + הגדלת מכירות + שיפור שביעות רצון (בערך כסף)
    \item \textbf{עלות כוללת} = מנויים + הטמעה + הדרכה + תחזוקה
\end{itemize}
\end{hebrew}

\subsection{נוסחת עלות טוקנים}
\label{subsec:token_cost}

\begin{hebrew}
להבנת העלות התפעולית של שימוש ב-\textenglish{API}:
\end{hebrew}

\begin{equation}
\hebmath{עלות שיחה} = \left(\frac{\hebmath{טוקני קלט}}{1{,}000{,}000} \times \hebmath{מחיר קלט}\right) + \left(\frac{\hebmath{טוקני פלט}}{1{,}000{,}000} \times \hebmath{מחיר פלט}\right)
\label{eq:token_cost}
\end{equation}

\begin{hebrew}
\textbf{דוגמה:}

סוכן \textenglish{AI} לניתוח משוב לקוחות משתמש ב-\textenglish{GPT-4 Turbo}:
\begin{itemize}
    \item מחיר קלט: \textenglish{\$10} למיליון טוקנים
    \item מחיר פלט: \textenglish{\$30} למיליון טוקנים
    \item כל ניתוח כולל: \textenglish{3,000} טוקני קלט + \textenglish{1,200} טוקני פלט
    \item נפח: \textenglish{5,000} ניתוחים בחודש
\end{itemize}

עלות לניתוח בודד:
\begin{equation}
\begin{split}
\text{Cost}_{\text{single}} &= \left(\frac{3{,}000}{1{,}000{,}000} \times 10\right) + \left(\frac{1{,}200}{1{,}000{,}000} \times 30\right) \\
&= 0.03 + 0.036 \\
&= \$0.066
\end{split}
\end{equation}

עלות חודשית:
\begin{equation}
\text{Cost}_{\text{monthly}} = 5{,}000 \times 0.066 = \$330
\end{equation}

\textbf{שימוש מעשי:} נוסחה זו מאפשרת למנהלים לחזות עלויות על בסיס נפח העסקאות הצפוי ולהחליט בין מודלים שונים.
\end{hebrew}

\subsection{\textenglish{Break-Even Analysis}}
\label{subsec:breakeven}

\begin{hebrew}
מתי כדאי לעבור משירות \textenglish{cloud} לפתרון \textenglish{self-hosted}?
\end{hebrew}

\begin{equation}
\hebmath{נקודת איזון (חודשים)} = \frac{\hebmath{עלות הקמה חד-פעמית}}{\hebmath{חיסכון חודשי}}
\end{equation}

\begin{hebrew}
\textbf{דוגמה:}

חברה משתמשת ב-\textenglish{GPT-4 API} ומשלמת \textenglish{\$5,000}/חודש. היא שוקלת להקים שרת פנימי עם \textenglish{Llama 3 70B}.

\textbf{עלויות \textenglish{self-hosted}:}
\begin{itemize}
    \item שרת + \textenglish{GPU}: \textenglish{\$15,000}
    \item הקמה ותצורה: \textenglish{\$5,000}
    \item עלות חודשית (חשמל, תחזוקה, כ"א): \textenglish{\$2,000}
\end{itemize}

חיסכון חודשי:
\begin{equation}
\hebmath{חיסכון} = 5{,}000 - 2{,}000 = \$3{,}000
\end{equation}

נקודת איזון:
\begin{equation}
\text{Break-even} = \frac{15{,}000 + 5{,}000}{3{,}000} = 6.67 \hebmath{ חודשים}
\end{equation}

\textbf{משמעות:} תוך כ-\textenglish{7} חודשים ההשקעה תשתלם, ומשם והלאה החברה תחסוך \textenglish{\$3,000}/חודש.
\end{hebrew}

\section{השוואת מודלים: תרשים עלות מול יכולות}
\label{sec:model_comparison}

\begin{figure}[htbp]
\centering
\begin{english}
\begin{tikzpicture}
    \begin{axis}[
        xlabel={\he{עלות יחסית (למיליון טוקנים)}},
        ylabel={\he{יכולת (ציון ממוצע)}},
        xmin=0, xmax=12,
        ymin=60, ymax=100,
        xtick={0,2,4,6,8,10,12},
        ytick={60,70,80,90,100},
        legend pos=north west,
        ymajorgrids=true,
        xmajorgrids=true,
        grid style=dashed,
        width=12cm,
        height=8cm,
    ]

    % GPT-4 Turbo
    \addplot[
        color=blue,
        mark=*,
        mark size=3pt,
    ]
    coordinates {
        (10,95)
    };
    \addlegendentry{\textenglish{GPT-4 Turbo}}

    % GPT-3.5 Turbo
    \addplot[
        color=green,
        mark=square*,
        mark size=3pt,
    ]
    coordinates {
        (0.5,78)
    };
    \addlegendentry{\textenglish{GPT-3.5 Turbo}}

    % Claude 3.5 Sonnet
    \addplot[
        color=red,
        mark=triangle*,
        mark size=3pt,
    ]
    coordinates {
        (3,92)
    };
    \addlegendentry{\textenglish{Claude 3.5 Sonnet}}

    % Claude 3 Haiku
    \addplot[
        color=orange,
        mark=diamond*,
        mark size=3pt,
    ]
    coordinates {
        (0.25,75)
    };
    \addlegendentry{\textenglish{Claude 3 Haiku}}

    % Gemini 1.5 Pro
    \addplot[
        color=purple,
        mark=pentagon*,
        mark size=3pt,
    ]
    coordinates {
        (1.25,88)
    };
    \addlegendentry{\textenglish{Gemini 1.5 Pro}}

    \end{axis}
\end{tikzpicture}
\end{english}
\caption{\he{השוואת מודלים מובילים - עלות מול יכולות}}
\label{fig:model_comparison}
\end{figure}

\begin{hebrew}
תרשים \ref{fig:model_comparison} מציג את הטרייד-אוף בין עלות ליכולות:
\begin{itemize}
    \item \textbf{\textenglish{GPT-4 Turbo}:} היקר והמסוגל ביותר - מתאים למשימות קריטיות ומורכבות
    \item \textbf{\textenglish{Claude 3.5 Sonnet}:} איזון מצוין - 92\% מהיכולת ב-30\% מהמחיר
    \item \textbf{\textenglish{Gemini 1.5 Pro}:} חלופה חזקה במחיר נוח
    \item \textbf{\textenglish{GPT-3.5} ו-\textenglish{Haiku}:} למשימות פשוטות בנפח גבוה
\end{itemize}

\textbf{אסטרטגיה מומלצת:} שימוש במודלים שונים לצרכים שונים. משימות פשוטות (סיכום, עריכה) במודלים זולים; משימות מורכבות (ניתוח, החלטות) במודלים מתקדמים.
\end{hebrew}

\section{תרשים \textenglish{Venn}: חפיפה בין יכולות אנושיות ו-\textenglish{AI}}
\label{sec:human_ai_venn}

\begin{figure}[htbp]
\centering
\begin{english}
\begin{tikzpicture}[scale=1.2]
    % Left circle - Human (English only for TikZ)
    \draw[thick, fill=blue!20, fill opacity=0.5] (0,0) circle (2.5cm);
    \node at (-2,2) {\textbf{Human Only}};
    \node[align=center, font=\small] at (-2,1.2) {Deep Empathy};
    \node[align=center, font=\small] at (-2,0.6) {Intuition};
    \node[align=center, font=\small] at (-2,0) {Revolutionary Creativity};
    \node[align=center, font=\small] at (-2,-0.6) {Emotional Understanding};
    \node[align=center, font=\small] at (-2,-1.2) {Ethical Judgment};

    % Right circle - AI
    \draw[thick, fill=green!20, fill opacity=0.5] (3,0) circle (2.5cm);
    \node at (5,2) {\textbf{AI Only}};
    \node[align=center, font=\small] at (5,1.2) {Massive Data Processing};
    \node[align=center, font=\small] at (5,0.6) {24/7 Availability};
    \node[align=center, font=\small] at (5,0) {Full Consistency};
    \node[align=center, font=\small] at (5,-0.6) {Processing Speed};
    \node[align=center, font=\small] at (5,-1.2) {Neutrality (No Fatigue)};

    % Intersection
    \node at (1.5,2.5) {\textbf{Overlap}};
    \node[align=center, font=\small] at (1.5,1) {Text Analysis};
    \node[align=center, font=\small] at (1.5,0.4) {Translation};
    \node[align=center, font=\small] at (1.5,-0.2) {Summarization};
    \node[align=center, font=\small] at (1.5,-0.8) {Content Writing};
    \node[align=center, font=\small] at (1.5,-1.4) {Categorization};
\end{tikzpicture}
\end{english}
\caption{\he{חפיפה בין יכולות אנושיות ויכולות AI}}
\label{fig:venn_diagram}
\end{figure}

\begin{hebrew}
תרשים \ref{fig:venn_diagram} ממחיש היכן כדאי להשתמש ב-\textenglish{AI}, והיכן האדם עדיין בלתי תחליפי:

\textbf{אזור החפיפה} - משימות שבהן \textenglish{AI} יכול לסייע או אפילו להחליף אדם:
\begin{itemize}
    \item עיבוד ואנליזת טקסט
    \item תרגום והתאמה לשונית
    \item סיכום מסמכים ארוכים
    \item כתיבת תוכן סטנדרטי
    \item מיון וקטגוריזציה
\end{itemize}

\textbf{אזור אנושי בלבד} - משימות שבהן האדם עדיין חיוני:
\begin{itemize}
    \item החלטות אסטרטגיות מורכבות
    \item מצבים הדורשים אמפתיה אמיתית
    \item יצירתיות פורצת דרך (לא הרכבה של דפוסים)
    \item שיקול דעת אתי ומוסרי
    \item הבנה עמוקה של הקשר ארגוני ותרבותי
\end{itemize}

\textbf{אזור \textenglish{AI} בלבד} - משימות שבהן \textenglish{AI} עדיף על אדם:
\begin{itemize}
    \item עיבוד נפחי מידע עצומים
    \item זמינות מתמדת ללא הפסקה
    \item עקביות מוחלטת בביצוע
    \item מהירות ותגובה מיידית
    \item ניטרליות (אין השפעת עייפות, רגשות חולפים)
\end{itemize}

\textbf{המסקנה המנהלית:} הגישה האופטימלית היא \textbf{שיתוף פעולה}, לא תחרות. \textenglish{AI} כעוזר שמשחרר את האדם ממשימות שגרתיות, ומאפשר לו להתמקד ביתרונות הייחודיים שלו.
\end{hebrew}

\section{דוגמאות מעשיות: \textenglish{LLMs} בעבודה}
\label{sec:practical_examples}

\subsection{מנהלת שיווק: תכנון ויצירת קמפיין}
\label{subsec:example_marketing}

\begin{hebrew}
\textbf{תרחיש:}

רינה, מנהלת שיווק בחברת \textenglish{SaaS}, צריכה להשיק קמפיין למוצר חדש. לפניה:
\begin{itemize}
    \item בניית אסטרטגיה
    \item כתיבת תוכן לערוצים שונים
    \item יצירת וריאציות למבחני \textenglish{A/B}
\end{itemize}

\textbf{השימוש ב-\textenglish{ChatGPT}:}

\textbf{שלב 1 - סיעור מוחות אסטרטגי:}
\begin{quote}
"אני משיקה מוצר \textenglish{SaaS} לניהול פרויקטים לצוותי שיווק. קהל יעד: מנהלי שיווק ב-\textenglish{B2B}. הערך המרכזי: חיסכון של \textenglish{10} שעות שבועיות באוטומציה. תן לי \textenglish{5} זוויות שונות לקמפיין."
\end{quote}

\textenglish{ChatGPT} מציע:
\begin{enumerate}
    \item "עשה יותר עם פחות" - דגש על יעילות
    \item "מזמן לאסטרטגיה" - שחרור מדיווחים לחשיבה
    \item "צוות קטן, תוצאות גדולות" - יתרון לסטארטאפים
    \item "\textenglish{ROI} ברור מיום ראשון"
    \item "כל הכלים במקום אחד" - פשטות
\end{enumerate}

\textbf{שלב 2 - פיתוח זווית נבחרת:}

רינה בחרה בזווית \textenglish{2}. היא ממשיכה:
\begin{quote}
"תפתח את זווית 'מזמן לאסטרטגיה'. כתוב לי \textenglish{3} כותרות ל-\textenglish{LinkedIn Ads}, \textenglish{2} פוסטים קצרים, ודף נחיתה בסגנון חברי-מקצועי."
\end{quote}

תוך דקה וחצי, רינה מקבלת תוכן מלא. היא משקיעה עוד \textenglish{20} דקות בעריכה ובהתאמה למותג.

\textbf{שלב 3 - וריאציות ל-\textenglish{A/B Testing}:}
\begin{quote}
"צור עוד \textenglish{5} וריאציות של הכותרת הראשונה - אחת עם מספרים, אחת עם שאלה, אחת עם אמוג'י, אחת קצרה מאוד, ואחת ארוכה ומפורטת."
\end{quote}

\textbf{תוצאה:}
\begin{itemize}
    \item זמן שנחסך: \textenglish{4-5} שעות של כתיבה וחשיבה
    \item איכות: טובה מאוד (לאחר עריכה קלה)
    \item יצירתיות: זוויות שרינה לא חשבה עליהן
\end{itemize}
\end{hebrew}

\subsection{סמנכ"ל כספים: ניתוח דוחות מורכבים}
\label{subsec:example_cfo}

\begin{hebrew}
\textbf{תרחיש:}

יוסי, \textenglish{CFO} של חברה ציבורית, מקבל דוח רבעוני של \textenglish{80} עמודים מחברת ביקורת. הוא צריך:
\begin{itemize}
    \item להבין מגמות עיקריות
    \item לזהות חריגות
    \item להכין סיכום להנהלה
\end{itemize}

\textbf{השימוש ב-\textenglish{Claude} (יכולת העלאת קבצים):}

יוסי מעלה את ה-\textenglish{PDF} ל-\textenglish{Claude} ושואל:
\begin{quote}
"נתח את הדוח הזה וספק:
\textenglish{1}. \textenglish{3} המגמות החיוביות העיקריות
\textenglish{2}. \textenglish{3} נקודות הדאגה העיקריות
\textenglish{3}. חריגות משמעותיות מול רבעון קודם
\textenglish{4}. המלצות ראשוניות"
\end{quote}

\textenglish{Claude} קורא את כל הדוח (למרות אורכו) ותוך \textenglish{2} דקות מספק:

\textbf{מגמות חיוביות:}
\begin{enumerate}
    \item צמיחה של 12\% בהכנסות חוזרות (\textenglish{ARR})
    \item שיפור של 5\% ב-\textenglish{Gross Margin}
    \item ירידה של 18\% בעלויות רכישת לקוח (\textenglish{CAC})
\end{enumerate}

\textbf{נקודות דאגה:}
\begin{enumerate}
    \item עלייה של 22\% ב-\textenglish{churn rate} בסגמנט SMB
    \item הארכת מחזור המכירה מ-\textenglish{45} ל-\textenglish{63} ימים
    \item עלייה בהוצאות \textenglish{R\&D} ללא השקה מקבילה
\end{enumerate}

\textbf{חריגות:}
\begin{itemize}
    \item סעיף "הוצאות שונות" קפץ פי \textenglish{3} - דורש הבהרה
    \item נכס בלתי מוחשי חדש בסכום חריג
\end{itemize}

\textbf{המלצות:}
\begin{itemize}
    \item לחקור את עליית ה-\textenglish{churn} - אפשרי שינוי בתחרות או בטיב שירות
    \item לברר את הארכת מחזור המכירה - האם בעיה במכירות או בכלכלה הכללית
    \item לדרוש פירוט על "הוצאות שונות"
\end{itemize}

יוסי עובר על הניתוח, משווה למסמכים נוספים, ומעמיק בנקודות הקריטיות. אבל הסינון הראשוני חסך לו שעה וחצי של קריאה צפופה.

\textbf{שימוש נוסף - שאלות המשך:}

יוסי ממשיך לשאול:
\begin{quote}
"על בסיס הדוח, האם נוכל לעמוד ביעדי ה-\textenglish{EBITDA} לסוף השנה?"
\end{quote}

\textenglish{Claude} מנתח את הנתונים ומספק תשובה מנומקת עם הנחות.
\end{hebrew}

\subsection{מנהלת משאבי אנוש: סינון קורות חיים}
\label{subsec:example_hr}

\begin{hebrew}
\textbf{תרחיש:}

מיכל, מנהלת גיוס, קיבלה \textenglish{150} קורות חיים לתפקיד \textenglish{Product Manager}. דרישות התפקיד:
\begin{itemize}
    \item ניסיון של \textenglish{3-5} שנים בניהול מוצר \textenglish{B2B SaaS}
    \item רקע טכני - יתרון למי שעבד כמפתח
    \item ניסיון בעבודה עם צוותים מבוזרים
    \item אנגלית שוטפת
\end{itemize}

\textbf{גישה מסורתית:} מיכל הייתה צריכה לקרוא כל קו"ח ידנית - זמן: \textenglish{5-6} שעות.

\textbf{גישה \textenglish{AI}:}

מיכל כותבת סקריפט \textenglish{Python} פשוט (בעזרת \textenglish{ChatGPT}) שקורא את כל הקבצים ושולח כל קו"ח ל-\textenglish{GPT-4} עם ה-\textenglish{prompt}:

\begin{quote}
"דרג קורות חיים אלה לתפקיד Product Manager ב-B2B SaaS. דרישות:
- \textenglish{3-5} שנות ניסיון \textenglish{PM}
- רקע טכני (יתרון)
- עבודה עם צוותים מבוזרים
- אנגלית שוטפת

דרג: STRONG FIT / GOOD FIT / MAYBE / NO FIT
ונמק בקצרה."
\end{quote}

המערכת מעבדת את \textenglish{150} הקורות חיים תוך \textenglish{10} דקות ומחזירה:
\begin{itemize}
    \item \textenglish{12} מועמדים \textenglish{STRONG FIT}
    \item \textenglish{28} מועמדים \textenglish{GOOD FIT}
    \item \textenglish{45} מועמדים \textenglish{MAYBE}
    \item \textenglish{65} מועמדים \textenglish{NO FIT}
\end{itemize}

מיכל קוראת רק את \textenglish{12} ה-\textenglish{STRONG FIT} ואת \textenglish{28} ה-\textenglish{GOOD FIT} - סך הכל \textenglish{40} קורות חיים - זמן: שעה.

\textbf{תוצאה:} חיסכון של \textenglish{4-5} שעות, תוך שמירה על איכות הסינון.

\textbf{הערה חשובה:} מיכל לא סומכת באופן עיוור על ה-\textenglish{AI}. היא עדיין קוראת בעצמה את הקורות חיים הרלוונטיים. ה-\textenglish{AI} משמש ככלי סינון ראשוני, לא כקובע סופי.
\end{hebrew}

\section{תרגילים}
\label{sec:exercises}

\subsection{תרגילים תיאורטיים}
\label{subsec:theoretical_exercises}

\begin{exercise}
\textbf{חישוב \textenglish{ROI} של הטמעת כלי \textenglish{AI}}

הנך מנהל/ת מחלקת תוכן בחברת \textenglish{e-commerce}. הצוות שלך (\textenglish{8} כותבים) מייצר \textenglish{40} מאמרים בחודש. כל מאמר לוקח כיום \textenglish{4} שעות עבודה.

אתה שוקל להטמיע \textenglish{ChatGPT Plus} (\textenglish{\$20}/חודש לכל כותב) שלדעתך יקצר את הזמן ל-\textenglish{2.5} שעות למאמר.

\textbf{נתונים:}
\begin{itemize}
    \item עלות שעת עבודה ממוצעת: \textenglish{120} ש"ח
    \item ימי עבודה בחודש: \textenglish{22}
\end{itemize}

\textbf{שאלות:}
\begin{enumerate}[label=\alph*)]
    \item חשב את ה-\textenglish{ROI} החודשי
    \item כמה זמן יעבור עד שההשקעה תשתלם (בהנחה שיש גם עלות הדרכה חד-פעמית של \textenglish{5,000} ש"ח)?
    \item האם תמליץ על ההטמעה? נמק.
\end{enumerate}
\end{exercise}

\begin{exercise}
\textbf{זיהוי משימות מתאימות ובלתי מתאימות ל-\textenglish{LLM}}

בחן את המשימות הבאות בארגון שלך וסווג כל אחת:
\begin{itemize}
    \item א. מתאימה מאוד ל-\textenglish{LLM} (יכול להחליף אדם לחלוטין)
    \item ב. מתאימה חלקית (יכול לסייע לאדם)
    \item ג. לא מתאימה (האדם חיוני)
\end{itemize}

\textbf{משימות:}
\begin{enumerate}
    \item כתיבת מדיניות פרטיות לאתר
    \item קבלת החלטה על פיטורי עובד
    \item תרגום חוזה מאנגלית לעברית
    \item בניית אסטרטגיה עסקית ל-\textenglish{3} שנים
    \item סיכום פרוטוקול ישיבה
    \item ניהול משא ומתן עם לקוח כועס
    \item ניתוח משוב לקוחות מסקר (\textenglish{NPS})
    \item בחירת ספק אסטרטגי לטווח ארוך
    \item יצירת תבניות מייל לתמיכת לקוחות
    \item ראיון עומק עם מועמד לתפקיד בכיר
\end{enumerate}

\textbf{נמק את הבחירות שלך.}
\end{exercise}

\begin{exercise}
\textbf{השוואה בין \textenglish{GPT-4} ל-\textenglish{Claude} לצורכי הארגון}

הנך \textenglish{CTO} של חברת \textenglish{FinTech}. אתה צריך לבחור מודל \textenglish{LLM} מרכזי לארגון.

\textbf{תרחישי שימוש מרכזיים:}
\begin{itemize}
    \item ניתוח מסמכים משפטיים ופיננסיים (דוחות, חוזים)
    \item תמיכת לקוחות אוטומטית
    \item סיוע למפתחים בכתיבת קוד
    \item יצירת תוכן שיווקי
\end{itemize}

\textbf{קריטריונים:}
\begin{itemize}
    \item עלות (נפח גבוה - \textenglish{50} מיליון טוקנים/חודש)
    \item דיוק במסמכים ארוכים
    \item יכולת קוד
    \item תמיכה בעברית
    \item אבטחה ופרטיות
\end{itemize}

\textbf{בנה טבלת השוואה ובחר מודל. נמק.}
\end{exercise}

\begin{exercise}
\textbf{ניתוח כישלון \textenglish{AI} - מה השתבש?}

חברת ביטוח הטמיעה סוכן \textenglish{AI} לתמיכת לקוחות. אחרי חודש, היא גילתה שהסוכן:
\begin{itemize}
    \item נתן מידע שגוי על כיסוי ביטוחי ב-15\% מהמקרים
    \item "המציא" פוליסות שלא קיימות
    \item לא הצליח להבין שאלות בעברית עם מונחים ביטוחיים
\end{itemize}

\textbf{שאלות:}
\begin{enumerate}[label=\alph*)]
    \item מה הסיבות האפשריות לכישלון?
    \item איזה חולשות של \textenglish{LLMs} באו לידי ביטוי?
    \item איך היית ממליץ לתקן את המערכת?
    \item האם היית ממליץ להפסיק את הפרויקט לחלוטין? למה?
\end{enumerate}
\end{exercise}

\begin{exercise}
\textbf{בניית \textenglish{Business Case} ליישום \textenglish{LLM}}

הנך מנהל/ת מחלקת מכירות עם \textenglish{20} נציגי מכירות. אתה רוצה להטמיע סוכן \textenglish{AI} שיסייע להם במשימות הבאות:
\begin{itemize}
    \item כתיבת הצעות מחיר מותאמות אישית
    \item מענה על שאלות טכניות של לקוחות (באמצעות \textenglish{RAG} על מסמכי המוצר)
    \item סיכום שיחות עם לקוחות ומעקב אוטומטי
\end{itemize}

\textbf{כתוב \textenglish{Business Case} שכולל:}
\begin{enumerate}[label=\alph*)]
    \item \textbf{בעיה עסקית:} מה הכאב הנוכחי?
    \item \textbf{פתרון מוצע:} מה בדיוק תטמיע?
    \item \textbf{תועלת:} מה יהיו היתרונות המדידים?
    \item \textbf{עלויות:} כמה זה יעלה? (כולל טכנולוגיה, הטמעה, הדרכה)
    \item \textbf{\textenglish{ROI}:} תוך כמה זמן ההשקעה תשתלם?
    \item \textbf{סיכונים:} מה עלול להשתבש?
    \item \textbf{המלצה:} האם כדאי להתקדם?
\end{enumerate}
\end{exercise}

\subsection{תרגילי קוד \textenglish{Python}}
\label{subsec:coding_exercises}

\begin{exercise}
\textbf{חישוב עלות שימוש חודשית ב-\textenglish{API}}

כתוב תוכנית \textenglish{Python} שמקבלת:
\begin{itemize}
    \item מספר שיחות/פעולות חודשיות
    \item ממוצע טוקני קלט לפעולה
    \item ממוצע טוקני פלט לפעולה
    \item מחיר קלט למיליון טוקנים
    \item מחיר פלט למיליון טוקנים
\end{itemize}

\textbf{התוכנית צריכה לחשב ולהדפיס:}
\begin{enumerate}[label=\alph*)]
    \item עלות לפעולה בודדת
    \item עלות חודשית כוללת
    \item עלות שנתית
    \item השוואה: כמה היה עולה באותו נפח עם מודל אחר (שהמשתמש מזין מחירים שלו)
\end{enumerate}

\textbf{דוגמת הרצה:}
\begin{latin}
\begin{verbatim}
Enter monthly operations: 50000
Enter avg input tokens: 2000
Enter avg output tokens: 800
Enter input price ($/1M tokens): 10
Enter output price ($/1M tokens): 30

=== Cost Analysis ===
Cost per operation: $0.044
Monthly cost: $2,200.00
Annual cost: $26,400.00

Compare with another model? (y/n): y
Enter input price ($/1M tokens): 0.5
Enter output price ($/1M tokens): 1.5

Alternative model monthly cost: $112.00
You would save: $2,088.00/month (94.9%)
\end{verbatim}
\end{latin}
\end{exercise}

\section{סיכום הפרק}
\label{sec:chapter_summary}

\begin{hebrew}
במהלך פרק זה עברנו מסע מקיף בעולם מודלי השפה הגדולים:

\textbf{למדנו מהם \textenglish{LLMs}:}
\begin{itemize}
    \item מכונות השלמת דפוסים שאומנו על כמויות עצומות של טקסט
    \item לא מסדי נתונים, אלא מודלים סטטיסטיים מורכבים שיוצרים תוכן חדש
\end{itemize}

\textbf{הכרנו את שני הכוחות העל:}
\begin{itemize}
    \item \textbf{תקשורת טבעית:} סוף סוף, מכונות שמבינות אותנו
    \item \textbf{עיבוד לוגיקה מורכבת:} לא רק ממשק נוח, אלא חשיבה אמיתית
\end{itemize}

\textbf{זיהינו נקודות חוזק:}
\begin{itemize}
    \item יצירתיות והפקת רעיונות
    \item הבנת הקשר ונואנסים
    \item גמישות והתאמה
    \item יכולת רב-לשונית
\end{itemize}

\textbf{למדנו על נקודות חולשה קריטיות:}
\begin{itemize}
    \item הזיות - המצאת מידע שנשמע מהימן
    \item חוסר עדכניות - מוגבל לתאריך אימון
    \item עלויות - לא זניח בקנה מידה
    \item חוסר שקיפות - קופסה שחורה
\end{itemize}

\textbf{רכשנו כלים מנהליים:}
\begin{itemize}
    \item נוסחת \textenglish{ROI} להערכת השקעה
    \item נוסחת עלות טוקנים לתכנון תקציב
    \item ניתוח \textenglish{break-even} להשוואת אלטרנטיבות
\end{itemize}

\textbf{ראינו דוגמאות מהשטח:}
\begin{itemize}
    \item מנהלת שיווק שחוסכת שעות ביצירת תוכן
    \item \textenglish{CFO} שמנתח דוחות מורכבים במהירות
    \item מנהלת \textenglish{HR} שמסננת קורות חיים ביעילות
\end{itemize}

\textbf{המסר המרכזי:}

\textenglish{LLMs} הם לא קסם, והם לא מושלמים. אבל כשמבינים את היכולות והמגבלות שלהם, ומיישמים אותם בצורה מושכלת - הם כלי עסקי עוצמתי שמשנה משחק.

ההצלחה לא תלויה בטכנולוגיה בלבד, אלא באופן שבו מנהלים מבינים, מתכננים ומיישמים אותה. פרק זה סיפק לכם את היסודות - בפרקים הבאים נצלול עמוק יותר לאקוסיסטם, לארכיטקטורה, וליישום מעשי.
\end{hebrew}

\section*{מקורות והמלצות לקריאה נוספת}
\begin{hebrew}
\begin{enumerate}
    \item \textenglish{Attention Is All You Need} - מאמר היסוד על ארכיטקטורת \textenglish{Transformer}
    \item \textenglish{OpenAI GPT-4 Technical Report} - תיעוד רשמי של \textenglish{GPT-4}
    \item \textenglish{Anthropic Claude Documentation} - מדריך מקיף למודלי \textenglish{Claude}
    \item \textenglish{The Economics of Large Language Models} - ניתוח עלויות ו-\textenglish{ROI}
    \item \textenglish{Prompt Engineering Guide} - מדריך מעמיק לכתיבת \textenglish{prompts}
\end{enumerate}
\end{hebrew}

\section*{פתרונות מלאים לתרגילים}
\label{sec:solutions}

\subsection*{פתרון תרגיל 1: חישוב \textenglish{ROI}}
\begin{hebrew}
\textbf{נתונים:}
\begin{itemize}
    \item \textenglish{8} כותבים
    \item \textenglish{40} מאמרים/חודש
    \item זמן נוכחי: \textenglish{4} שעות/מאמר
    \item זמן עתידי: \textenglish{2.5} שעות/מאמר
    \item עלות שעה: \textenglish{120} ש"ח
    \item עלות \textenglish{ChatGPT Plus}: \textenglish{\$20}/חודש = ~\textenglish{75} ש"ח
\end{itemize}

\textbf{חישוב:}

סך שעות נוכחי:
\begin{equation}
40 \times 4 = 160 \hebmath{ שעות/חודש}
\end{equation}

סך שעות עתידי:
\begin{equation}
40 \times 2.5 = 100 \hebmath{ שעות/חודש}
\end{equation}

שעות נחסכות:
\begin{equation}
160 - 100 = 60 \hebmath{ שעות/חודש}
\end{equation}

ערך החיסכון:
\begin{equation}
60 \times 120 = 7{,}200 \hebmath{ ש"ח/חודש}
\end{equation}

עלות מנוי:
\begin{equation}
8 \times 75 = 600 \hebmath{ ש"ח/חודש}
\end{equation}

\textenglish{ROI} חודשי:
\begin{equation}
\frac{7{,}200 - 600}{600} \times 100\% = 1{,}100\%
\end{equation}

\textbf{זמן החזר השקעה:}

חיסכון נטו חודשי:
\begin{equation}
7{,}200 - 600 = 6{,}600 \hebmath{ ש"ח}
\end{equation}

עלות הדרכה:
\begin{equation}
5{,}000 \hebmath{ ש"ח}
\end{equation}

זמן החזר:
\begin{equation}
\frac{5{,}000}{6{,}600} = 0.76 \hebmath{ חודשים} \approx 23 \hebmath{ ימים}
\end{equation}

\textbf{המלצה:} בהחלט כדאי! ה-\textenglish{ROI} עצום, וההשקעה מתשלמת תוך פחות מחודש.
\end{hebrew}

\subsection*{פתרון תרגיל 2: זיהוי משימות}
\begin{hebrew}
\begin{enumerate}
    \item \textbf{כתיבת מדיניות פרטיות:} ב' (מתאים חלקית) - \textenglish{AI} יכול לכתוב טיוטה מצוינת, אבל עורך דין צריך לאשר.

    \item \textbf{החלטה על פיטורים:} ג' (לא מתאים) - דורש שיקול דעת אנושי, אמפתיה, הבנה של הקשר ארגוני.

    \item \textbf{תרגום חוזה:} ב' (מתאים חלקית) - תרגום ראשוני מצוין, אבל חוזה דורש עריכה משפטית אנושית.

    \item \textbf{אסטרטגיה ל-\textenglish{3} שנים:} ב'-ג' - \textenglish{AI} יכול לסייע בניתוח וברעיונות, אבל החלטה סופית דורשת אדם.

    \item \textbf{סיכום פרוטוקול:} א' (מתאים מאוד) - משימה מובנית שה-\textenglish{AI} עושה מצוין.

    \item \textbf{משא ומתן עם לקוח כועס:} ג' (לא מתאים) - דורש אמפתיה אמיתית ושיקול דעת דינמי.

    \item \textbf{ניתוח סקר \textenglish{NPS}:} א'-ב' (מתאים מאוד) - \textenglish{AI} מצוין בזיהוי דפוסים ומגמות.

    \item \textbf{בחירת ספק אסטרטגי:} ב'-ג' - \textenglish{AI} יכול לנתח, אבל החלטה כזו דורשת שיקולים רבים שאדם מבין טוב יותר.

    \item \textbf{תבניות מייל:} א' (מתאים מאוד) - משימה שה-\textenglish{AI} עושה מצוין.

    \item \textbf{ראיון לתפקיד בכיר:} ג' (לא מתאים) - דורש הבנה עמוקה של תרבות ארגונית ושיקול דעת אנושי.
\end{enumerate}
\end{hebrew}

\subsection*{פתרון תרגיל 6: קוד Python לחישוב עלויות}
\begin{hebrew}
להלן קוד מלא ומתועד:
\end{hebrew}

\begin{latin}
\begin{lstlisting}[language=Python, caption={מחשבון עלויות API למודלי LLM}]
"""
LLM API Cost Calculator
Calculates monthly and annual costs for LLM API usage
"""

def calculate_cost(operations, input_tokens, output_tokens,
                   input_price, output_price):
    """
    Calculate cost per operation and total costs

    Args:
        operations: Number of monthly operations
        input_tokens: Average input tokens per operation
        output_tokens: Average output tokens per operation
        input_price: Price per 1M input tokens ($)
        output_price: Price per 1M output tokens ($)

    Returns:
        dict: Cost breakdown
    """
    # Cost per operation
    cost_per_op = (
        (input_tokens / 1_000_000) * input_price +
        (output_tokens / 1_000_000) * output_price
    )

    # Total costs
    monthly_cost = operations * cost_per_op
    annual_cost = monthly_cost * 12

    return {
        'cost_per_operation': cost_per_op,
        'monthly_cost': monthly_cost,
        'annual_cost': annual_cost
    }

def compare_models(operations, input_tokens, output_tokens,
                   model1_prices, model2_prices):
    """
    Compare costs between two models

    Args:
        operations: Number of monthly operations
        input_tokens: Average input tokens
        output_tokens: Average output tokens
        model1_prices: (input_price, output_price) for model 1
        model2_prices: (input_price, output_price) for model 2

    Returns:
        dict: Comparison results
    """
    cost1 = calculate_cost(operations, input_tokens, output_tokens,
                          model1_prices[0], model1_prices[1])
    cost2 = calculate_cost(operations, input_tokens, output_tokens,
                          model2_prices[0], model2_prices[1])

    savings = cost1['monthly_cost'] - cost2['monthly_cost']
    savings_pct = (savings / cost1['monthly_cost']) * 100

    return {
        'model1_monthly': cost1['monthly_cost'],
        'model2_monthly': cost2['monthly_cost'],
        'monthly_savings': savings,
        'savings_percentage': savings_pct
    }

def main():
    """Main interactive calculator"""
    print("=== LLM API Cost Calculator ===\n")

    # Get user input
    operations = int(input("Enter monthly operations: "))
    input_tokens = int(input("Enter avg input tokens: "))
    output_tokens = int(input("Enter avg output tokens: "))
    input_price = float(input("Enter input price ($/1M tokens): "))
    output_price = float(input("Enter output price ($/1M tokens): "))

    # Calculate costs
    costs = calculate_cost(operations, input_tokens, output_tokens,
                          input_price, output_price)

    # Display results
    print("\n=== Cost Analysis ===")
    print(f"Cost per operation: ${costs['cost_per_operation']:.4f}")
    print(f"Monthly cost: ${costs['monthly_cost']:,.2f}")
    print(f"Annual cost: ${costs['annual_cost']:,.2f}")

    # Compare with another model?
    compare = input("\nCompare with another model? (y/n): ")
    if compare.lower() == 'y':
        alt_input_price = float(
            input("Enter input price ($/1M tokens): ")
        )
        alt_output_price = float(
            input("Enter output price ($/1M tokens): ")
        )

        comparison = compare_models(
            operations, input_tokens, output_tokens,
            (input_price, output_price),
            (alt_input_price, alt_output_price)
        )

        print(f"\nAlternative model monthly cost: "
              f"${comparison['model2_monthly']:,.2f}")
        print(f"You would save: "
              f"${comparison['monthly_savings']:,.2f}/month "
              f"({comparison['savings_percentage']:.1f}%)")

if __name__ == "__main__":
    main()
\end{lstlisting}
\end{latin}

\begin{hebrew}
\textbf{הרצת הקוד:}
\begin{itemize}
    \item שמור את הקוד בקובץ \texttt{llm\_cost\_calculator.py}
    \item הרץ: \texttt{python llm\_cost\_calculator.py}
    \item עקוב אחר ההנחיות האינטראקטיביות
\end{itemize}

\textbf{תרגיל מורחב:}
הרחב את הקוד כך שיתמוך ב:
\begin{enumerate}
    \item שמירת תוצאות לקובץ \textenglish{CSV}
    \item ויזואליזציה של השוואת עלויות (גרף)
    \item חישוב \textenglish{break-even} עבור מעבר בין מודלים
    \item תמיכה בהשוואה של יותר משני מודלים
\end{enumerate}
\end{hebrew}

\vspace{1cm}
\begin{center}
\rule{0.5\textwidth}{0.4pt}

\vspace{0.5cm}
\begin{hebrew}
\textbf{סוף הפרק הראשון}

בפרק הבא נצלול לאקוסיסטם הבינה המלאכותית - מפת הכלים, הספקים והטכנולוגיות שמנהל מודרני צריך להכיר.
\end{hebrew}
\end{center}

%% ============================================
%% Bibliography
%% ============================================
\begin{english}
\setlength{\bibitemsep}{0.5\baselineskip}
\printbibliography[title={\texthebrew{מקורות}}]
\end{english}

\end{document}
